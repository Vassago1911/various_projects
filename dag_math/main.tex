% This file is part of beamerthemepureminimalistic.

% If problems/bugs are found or enhancements are desired, please contact
% me over: https://github.com/kai-tub/latex-beamer-pure-minimalistic

\documentclass[a4paper]{book}
\include{version}
\usepackage{amsmath,amssymb,amsthm}
\newtheorem{thm}{Theorem}[section]
\newtheorem{theorem}{Theorem}[section]
\newtheorem{lem}[thm]{Lemma}
\newtheorem{lemma}[thm]{Lemma}
\newtheorem{prop}[thm]{Proposition}
\newtheorem{proposition}[thm]{Proposition}
\newtheorem{cor}[thm]{Corollary}
\newtheorem{corollary}[thm]{Corollary}
\theoremstyle{definition}
\newtheorem{Remi}[thm]{Reminder}
\newtheorem{rem}[thm]{Remark}
\newtheorem{defn}[thm]{Definition}
\newtheorem{ex}[thm]{Example}
\newtheorem{nonex}[thm]{{\bf NON}-Example}
\newtheorem{conj}[thm]{Conjecture}
\newtheorem{Exercise}[thm]{Exercise}
\newtheorem{Code}[thm]{Pseudocode}
\usepackage{datetime2}
\usepackage{graphicx}
\usepackage{svg}
% \svgpath{{/home/vassago/LATEX/}{/home/vassago/LATEX/svgs/}}
\usepackage{svg-extract}
\usepackage{hyperref}
\usepackage{inconsolata}

\usepackage{tcolorbox}

\usepackage[all]{xy}
\usepackage{xypic}
\usepackage{xcolor}

\definecolor{light}{RGB}{0,100,43}
\definecolor{dark}{RGB}{130,0,100}
\definecolor{exercise}{RGB}{220,200,255}
\definecolor{textcolor}{RGB}{0, 150, 128}
\definecolor{title}{RGB}{150, 0, 128}
\definecolor{footercolor}{RGB}{0, 128, 64}
\definecolor{code}{RGB}{240,220,160}
\definecolor{codebg}{RGB}{50,50,50}
\definecolor{mathbg}{RGB}{0,10,30}
\definecolor{mathtext}{RGB}{220,220,250}
\definecolor{bg}{RGB}{32, 0, 16}

\def\exercise#1{\color{light}{\h\Exercise{\color{exercise}\scshape #1}}\color{textcolor}}
\def\code#1#2{\color{dark}{\h\Code{\h}\\[.5em]%
\color{mathtext}#1
\begin{tcolorbox}[colback=codebg]%
\color{code}%
\texttt{#2}%
\color{textcolor}%
\end{tcolorbox}\color{textcolor}}}

\def\backtick{`}
\def\curtime{\DTMdate{2023-09-21} ~\DTMtime{01:42:17}\DTMdisplayzone{-5}{00}}
\def\light#1{{\color{light}#1}}
\def\dark#1{{\color{dark}#1}}
\def\h{\hspace{1em}}

\def\signature{\href{mailto:marc@lange-data.org}{Dr. Marc Lange, marc@lange-data.org}}
\def\mailsignature{\href{mailto:marc@lange-data.org}{marc@lange-data.org}}

\def\abs#1{\left| #1 \right| }
\def\floor#1{\lfloor #1 \rfloor }

\newcommand{\lthree}{<\hspace{-5.2pt}3}
\newcommand{\E}{\mathcal{E}}
\newcommand{\DAG}{\mathcal{DAG}}
\newcommand{\Z}{\mathbb{Z}}
\newcommand{\R}{\mathbb{R}}
\renewcommand{\P}{\mathfrak{P}}
\newcommand{\Q}{\mathbb{Q}}
\newcommand{\N}{\mathbb{N}}
\renewcommand{\S}{\mathbb{S}}
\newcommand{\C}{\mathcal{C}^{\leq}_{+} }
\newcommand{\D}{\mathbb{D}}
\newcommand{\LL}{\mathcal{L}^1(\Z,\Z)}
\newcommand{\Set}{\mathrm{Set}}

\renewcommand{\dag}{\mathrm{emb}\mathcal{DAG}}


\newcommand{\s}{Set^{\phantom{\leq}}_{\phantom{+}}}
\newcommand{\spp}{Set=fSet}
\newcommand{\smp}{Set^{\leq}_{\phantom{+}}}
\newcommand{\spm}{Set^{\phantom{\leq}}_{+}}
\newcommand{\smm}{Set^{\leq}_{+}}
\newcommand{\spmpm}{Set^{\pm}_{\pm}}

\newcommand{\App}{A}
\newcommand{\Amp}{A^{\leq}_{\phantom{+}}}
\newcommand{\Apm}{A^{\phantom{\leq}}_{+}}
\newcommand{\Amm}{A^{\leq}_{+}}

\begin{document}
\chapter{Counting DAGs}
\section{Basic Facts and Definitions}
I recall the definition of directed acyclic graphs informally here, in favour of providing a more
precise definition tailored to this paper's use case later.
\defn[Directed Acyclic Graph]{
    A directed graph $G = (V,E)$ with $E \subset V\times V$, is called acyclic, if there is no (finite) chain
    of vertices $(v_0,v_1,\ldots,v_n)$ with $v_i \in V, 1<n, (v_i,v_{i+1}) \in E$,
    such that at least one of the vertices $v_i$ has
    been visited twice:
    \[ G=(V,E\subset V\times V) ~~~ \mathrm{dag} \]
    \[\colon\leftrightarrow  \forall n\geq2 \forall (v_0,v_1,\ldots,v_n) \in V^{\times n} \colon ( \forall i\colon (v_i,v_{i+1}) \in E \Rightarrow \forall i\neq j\colon v_i\neq v_j ). \]

    Do note in particular that despite the fact that this is an infinite schema of axioms, it is finite for each specific finite digraph.
    More specifically, it could be coded to check an arbitrary digraph for dag-ness. In this text I will not use this however, our digraphs
    will come as dags plain by their code-friendly presentation.
}

\prop[Sources and Sinks exist]{
    \label{sources}
    Any finite dag has a vertex that has no outbound edges and a vertex that has no inbound edges.
    \begin{proof}
        Assume for contradiction that there is no vertex with no outbound edge, then choose any one of them, and append any next vertex,
        if there's an edge pointing to it from your current vertex. Since there's no sink, this process can continue infinitely, since
        our graph is finite, eventually we repeat a vertex. This is a contradiction to acyclicity as described above.

        The same argument applies to the digraph with inverted edges or just by walking the path backwards instead.
    \end{proof}
}

\defn[Topological Ordering]{
    If for a graph $G=(V,E)$ we have a graph homomorphism:
    \[ \varphi: G \rightarrow \Delta^{\abs{G}} = (\abs{G}=\{0,\ldots,\abs{G}\},\leq) \subset \N  \]
    which is injective on nodes, call $\varphi$ a topological ordering of $G$.

    More explicitly:
    If there exists a finite $n\in\N$ and a bijection $f \colon V \rightarrow \{0,\ldots,n\}$ such that
    $(i,j) \in E \Rightarrow fi \leq fj \in \N$ for all edges in $E$, then that bijection $f$ induces
    a topological ordering $\varphi$ like before.
}

\prop[Finite dags have topological orderings]{
    A digraph is a dag iff it has a topological ordering (necessarily non unique for non-trivial cases).
    \begin{proof}
        Assume a graph has a topological ordering, if it had a cycle, at least one edge would point backwards in that
        ordering, to contradiction. So the existence of a topological ordering implies the dag-ness of a (di)graph.

        Assume now an abstract dag $G = (V,E)$, we want to construct a bijection $\varphi\colon V\rightarrow \abs{V}$,
        such that $\varphi(i)\leq\varphi(j)$ whenever there is an edge $(i,j) \in E$ (recall: and no condition otherwise).

        By \hyperref[sources]{the fact that sources and sinks exist}, choose a source $s$. Since it has no inbound connections,
        set its label to $0$. Now remove $s$ from the graph $G$ with all edges that pointed from $s$ to $G$ and find that
        subgraph is a dag, too, so it has a source, that can be labelled $1$, and so forth up until all vertices have been
        exhausted.
    \end{proof}
}

\cor[Minimal topological orderings]{
    Each dag can be given a minimal topological ordering, that is, a bijection of its vertex set to an interval $\{0,\ldots,\abs{V}-1\}$
    that induces a graph homomorphism. (Corollary to the proof before.)
}

\section{The Category $\dag$ of embedded DAGs}

The section before discusses among other things that
each finite directed acyclic graph can be embedded in the complete finite directed acyclic graph of its vertex order.
Using that we reach the following definition for a combinatorial category of dags.

\defn[$\dag$]{
    Define the category $\dag$ as follows: \\
    As objects we have a chosen natural number $n\in\N$, call it vertex count, together with
    a chosen subset $E\subset \{0,\ldots,n-1\}\times\{0,\ldots,n-1\}$, which satisfies:
    \[ \forall (i,j) \in E\colon 0\leq i < j < n   ~~~~ (\mathrm{top}) , \]
    call $E$ the edge set of the object $(n,E)$.

    As morphisms we have all! set maps $f\colon \{0,\ldots,n-1\}\rightarrow \{0,\ldots,m-1\}$ for dags $(n,E),(m,F)$, such that:
    $\forall (i,j)\in E \colon (fi,fj) \in F$, i.e.
    \[ \dag((n,E),(m,F)) \]
    \[= \{ f\colon \{0,1,\ldots,n-1\} \rightarrow \{0,1,\ldots,m-1\} ~|~ (i,j) \in E \Rightarrow (fi,fj) \in F \vee fi=fj \}. \]
}


\rem{
    For clarity, the induced graph of such an object has vertex set $\{0,\ldots,n-1\}$ and edges as defined by $E$.
}

\rem{
    By the discussion above we can see, that the topological ordering has in fact been incorporated into this definition,
    so that the induced graph associated to a pair $(n,E)$ that satisfies the condition $\mathrm{top}$, is necessarily
    a dag. In particular in what follows there is no need to check the acyclicity condition, it is
    actually a given fact in $\dag$. Instead, in this category we need to keep compatible maps to $\N$ along.
}

\rem{
    One could instead choose the convention to consider dags with loops at each vertex, with would make
    the vertex set a recognisable subset of $E$, in which case the objects could be fully described by
    edge sets. Occassionally that might be a helpful extra perspective to consider, specifically when handling
    graph homomorphisms I jump between those conventions as convenient.
}

\rem{
    The category $\dag$ in particular admits an upper bound on counting how many dags there
    are on a fixed number of nodes. Do note from vertex count 3 on, isomorphic dags can have different
    representations in $\dag$, e.g. $\{0,1,2\},\{(0,1)\}$ and $\{0,1,2\},\{(0,2)\}$ are isomorphic,
    but each legitimately different objects in $\dag$.
}


\ex{
    The class of dags we need most instantly are complete graphs, which in accordance with topology I will call $\Delta^n$:
    \[V(\Delta^n) = \{ 0, 1, \ldots, n \}, E(\Delta^n) = \{ (i,j) ~|~ i<j \}.\]
    These are the edge count maximal objects in their vertex count, as in, there is no other dag on $n+1$ nodes with more or equally many edges to the $\Delta^n$ objects each in their degree.
}

\ex{
    For some more non-trivial examples:
     \[ \S^0 = ( \{0,1\},\{\} ), \]
     \[ \S^1 = ( \{0,1,2,3\},\{02,12,03,13\} ), \]
     \[ \S^2 = ( \{0,1,2,3,4,5\},\{02,12,03,13,04,14,24,34,05,15,25,35\} ). \]
    In a bit we will exhibit how these are actual spheres.

    It is however visually convincing:

    \[\S^0 =\xymatrix{0&1,}\]
    \[\S^1 =\xymatrix{ & 2 & \\ 0 \ar[dr]\ar[ur]& &1\ar[ul]\ar[dl] \\ & 3,&}\]
    \[\S^2 =\xymatrix{   &
                            &  & 4 &                                 \\
                         & 0\ar[d]\ar[ddr]\ar[urr]\ar[rrr]
                            &  &   & 2 \ar[ul] \ar[ddll]             \\
                         & 3 \ar[dr]\ar[uurr]
                            &  &   & 1 \ar[u]\ar[uul]\ar[dll]\ar[lll]\\
                         &
                            &5 &   & }\]
}

\defn[Cone and Suspension $C, \Sigma$]{
    In addition we have a cone and suspension construction:

    Let $G=(\{0,\ldots,n-1\},E)$, construct $CG$ with vertices $V(CG) = \{0,\ldots,n\}$
    and edges: $E(CG) = E(G) \sqcup \{ (i,n) | i \in \{0,\ldots,n-1 \} \}.$

    Let $G=(\{0,\ldots,n-1\},E)$, construct $\Sigma G$ with vertices $V(\Sigma G) = \{0,\ldots,n,n+1\}$
    and edges: $E(CG) = E(G) \sqcup \{ (i,n) | i \in \{0,\ldots,n-1 \} \} \sqcup \{ (i,n+1) | i \in \{0,\ldots,n-1 \} \}.$

    These are actual cones and suspensions as well.
}

\ex{
    For illustration consider the diagrams of $C\S^0$ and $\Sigma\S^0$:
    Start with $\S^0$:
    \[\xymatrix{0&1}\]
    add a cone point ($2$) and edges to it
    \[\xymatrix{0 \ar[r] & 2 & 1 \ar[l]}.\]
    For $\Sigma\S^0$ add another point $3$ and just not the edge $(2,3)$:
    \[\xymatrix{ & 2 & \\ 0 \ar[dr]\ar[ur]& &1\ar[ul]\ar[dl] \\ & 3.&}\]
}

\rem{
    More generally the illustrations for $C$ and $\Sigma$ are quite helpful:
    \[\xymatrix{ & C(pt) & \\ \bullet \ar[ur] \ar@{>.<}[rr] & & \ar[ul]\bullet, } \]
    \[\xymatrix{ & \Sigma^+(pt) & \\ \bullet \ar[dr]\ar[ur] \ar@{>.<}[rr] & & \ar[ul]\ar[dl] \bullet \\ & \Sigma^-(pt). &}\]
}


\ex{
    Do note we have:
    \[ C\Delta^{n-1} = \Delta^n ~\mathrm{for}~ n\geq -1. \]

    As well as:
    \[ \Sigma \S^{n-1} = \S^n \]
    for the examples defined above.
}

\defn[Simplices, Spheres, Disks $\Delta^n,\S^n,\D^n$]{
    Define $\Delta^-1 = (\emptyset,\emptyset)$, and inductively: $\Delta^n := C\Delta^{n-1}$ with $C$ the cone functor.
    Define $\S^0 = (\{0,1\},\emptyset)$, and inductively: $\S^n := \Sigma\S^{n-1}$.
    For $i\geq 1$ define: $\D^i = C\S^{i-1}$.

    Specifically for $i=1$ denote $I:=\D^1$ as our model for the unit interval in $\dag$.
}

For vertex count $1$ there are no edges to choose, so there is only one dag representing the one point
graph in the category $\dag$. For two vertices there is only one edge to choose, hence $2$ graphs
with vertex count $2$, which are non-isomorphic.

\prop[Number of legal incidence matrices in $\dag$]{
    For each degree $n\geq 0$, the count of distinct edge sets per degree can be calculated as:
    \[ \abs{ \dag_{0,n} } = 2^{{ n \choose 2 }}. \]
    \begin{proof}
        For a fixed degree $n\in\N$, we need to check, how many $E \subset \{0,\ldots,n-1\}\times\{0,\ldots,n-1\} $ satisfy the condition $\mathrm{top}$. In fact there is a maximal such $E$
        representing the complete dag $\Delta^{n-1}$ on $n$ vertices, satisfying:
        \[ (i,j) \in E(\Delta^{n-1}) \Leftrightarrow i<j. \]
        In particular each other such graph of vertex count $n$ can be reached by deleting a uniquely determined set of edges from $\Delta^{n-1}$.

        So count the edges of $\Delta^{n-1}$, since by definition they are each ordered, we can in fact
        count two-element subsets of $\{0,\ldots,{n-1}\}$ and interpret them canonically as an ordered
        tuple in each case. Hence the edge count is: \[\abs{E\Delta^{n-1}} = { n \choose 2 }. \]

        So the count of dags is identified as the number of (arbitrary) subsets of $E\Delta^{n-1}$:
        \[ \abs{ \dag_n }
        = 2^{\abs{E\Delta^{n-1}}}
        = 2^{{ n \choose 2 }}. \]
    \end{proof}
}

\cor{
    Do note in particular we have the recurrence relation:
    \[
    \abs{\dag_{0,n}} = \abs{\dag_{0,n-1}}*2^{n-1},
    \]
    so in particular we can interpret the formula through the
    sum over all natural numbers up to a threshold:
    \[
    \abs{\dag_{0,n}} = \prod_{i=0}^{n-1} 2^i = 2^{\sum_{i=0}^{n-1} i}.
    \]
    \begin{proof}
        By direct calculation with the above formula, we get:
        \[
        \begin{aligned}
        \frac{\abs{\dag_{0,n}}}{\abs{\dag_{0,n-1}}}=& 2^{{ n \choose 2 } - {n-1 \choose 2}},
        \end{aligned}
        \]
        and by the relation
        \[ { n \choose 2 } = { n-1 \choose 1 } + { n-1 \choose 2 } \]
        get
        \[
        \begin{aligned}
        \frac{\abs{\dag_{0,n}}}{\abs{\dag_{0,n-1}}}
        =& 2^{{ n-1 \choose 1 }}\\
        =& 2^{n-1}
        \end{aligned}
        \]
    \end{proof}
}

\rem{
    In particular one can see how extremely fast that count grows in $n$, essentially as
    $O(exp(n^2))$.

    \begin{tabular}{ l | l | r | r }
      vertex count & degree & dag count & log2 dag count\\
      \hline
     1 & 0 & 1 & 0 \\
     2 & 1 & 2 & 1 \\
     3 & 2 & 8 & 3 \\
     4 & 3 & 64 & 6 \\
     5 & 4 & 1024 & 10 \\
     6 & 5 & 32768 & 15 \\
     7 & 6 & 2097152 & 21 \\
     8 & 7 & 268435456 & 28 \\
     9 & 8 & 687194476736 & 36 \\
     10 & 9 & 35184372088832 & 45 \\
     11 & 10 & $\sim 3 * 10^{16}$ & 55 \\
     12 & 11 & $\sim 7 * 10^{19}$ & 66 \\
     13 & 12 & $\sim 3 * 10^{23}$ & 78 \\
    \end{tabular}

    In degree $23$ the count reaches $2^{276} \sim 10^{83},$ hence there are more
    such dags in degree $23$ in $\dag$, than there are atoms in the observable universe.
    At degree $14$ there are already about $10^{28}$ dags, hence more than atoms in a human body.
}

\lemma{
    There is a functor $\oplus\colon \dag \times \dag \rightarrow \dag$,
    given on objects and vertices as
    \[\{0,\ldots,n-1\}\oplus \{0,\ldots,m-1\} \colon= \{0,\ldots,n-1, n,\ldots,n+m-1\}, \]
    Given two dags $(n,E), (m,F)$, with the edge subsets:
    \[ (i,j) \in ( E\oplus F ) \Leftrightarrow (i,j) \in E \vee (i-n,j-n) \in F. \]
    Hence for two graph morphisms $\varphi\colon (n,E)\rightarrow (n',E'), \psi\colon (m,F)\rightarrow (m',F')$:
    \[ (\varphi \oplus \psi)(k) \colon= \begin{cases} \varphi(k)   & 0\leq k\leq n-1 \\
                                                     \psi(k-n)+n'  & n\leq k \leq n+m-1 \end{cases}. \]
}

\prop[Pushouts exist!]{
    WRONG!!! 02 -> 012, 02 -> 0 doesn't have a pushout

    The category of dags has (finite) pushouts!
    %todo: joins x*y und sigma(x*y) = smash of things
    \begin{proof}
    Assume given two dag-morphisms $\varphi_k\colon H\rightarrow G_k$, we want to construct an
    object $G_1\cup_HG_2$ and two morphisms $\iota_k\colon G_k \rightarrow G_1\cup_HG_2$, such
    that for each object $Z$ and pair of morphisms $f_k\colon G_k\rightarrow Z$ such that
    $f_1\circ\varphi_1 = f_2\circ\varphi_2$, there is a unique map $F\colon G_1\cup_HG_2 \rightarrow Z$,
    such that $F\circ\iota_k = f_k$.

    Assume $H = (\{0,\ldots,k-1\},E(H)), G_1 = (\{0,\ldots,l-1\},E(G_1)), G_2 = (\{0,\ldots,m-1\},E(G_2)),$
    and hence $\varphi_k\colon H \rightarrow G_k$ each represented by vertex maps
    $\varphi_1\colon \{0,\ldots,k-1\} \rightarrow \{0,\ldots,l-1\}$
    and $\varphi_2\colon \{0,\ldots,k-1\} \rightarrow \{0,\ldots,m-1\}$ respectively.
    \end{proof}
}

\rem{
    At this point I am fixing a bijection
    \[\omega_{n,m}\colon \{0,\ldots,n-1\}\times \{0,\ldots,m-1\} \rightarrow \{0,\ldots,nm-1\} \]
    which is appropriately symmetric and associative.

    Two choices of prominence are the idea of going through a 2d table either by row or by column
    first, like:
    \[ \omega_{n,m}(i,j) = i\cdot n + j, \]
    or
    \[ \bar{\omega}_{n,m}(i,j) = i + j\cdot m. \]
    These specifically satisfy:
    \[ \bar{\omega}_{n,m}(i,j) = \omega_{m,n}(j,i). \]
    There are actually at least countably infinitely many such choices,
    e.g. by choosing divisors of $n$ and $m$. Here I mainly want to choose one, sometimes use
    the other when convenient, but calm the reader that the specific choice is quite inessential.

    Occassionally I need the inverse, let us pick the one of $\omega_{n,m}$:
    \[\omega^{-1}_{n,m}(k) = ( \floor{k/n}, k~\mathrm{mod}~n ).\]
}

\ex{
\phantom{test}\\[2pt]
\begin{tabular}{l}
    { \begin{tabular}{l|r r r r r r r}
                    $\omega_{4,7}\colon$ & 0 & 1 & 2 & 3 & 4 & 5 & 6 \\ \hline
                    0 & 0 & 1 & 2 & 3 & 4 & 5 & 6 \\
                    1 & 7 & 8 & 9 &10 &11 &12 &13 \\
                    2 &14 &15 &16 &17 &18 &19 &20 \\
                    3 &21 &22 &23 &24 &25 &26 &27 \\
    \end{tabular} } \\
    \phantom{test}\\[2pt]
    {\begin{tabular}{l|r r r r r r r}
                    $\bar\omega_{4,7}\colon$ & 0 & 1 & 2 & 3 & 4 & 5 & 6 \\ \hline
                    0 & 0 & 4 & 8 &12 &16 &20 &24 \\
                    1 & 1 & 5 & 9 &13 &17 &21 &25 \\
                    2 & 2 & 6 &10 &14 &18 &22 &26 \\
                    3 & \phantom{0}3 & \phantom{0}7 &11 &15 &19 &23 &27 \\
    \end{tabular} }
\end{tabular}
}

\lemma{
    There is a functor $\otimes\colon \dag \times \dag \rightarrow \dag$,
    given on objects and vertices as
    \[\{0,\ldots,n-1\}\otimes \{0,\ldots,m-1\} \colon= \{0,\ldots,nm-1\}, \]
    Given two dags $(n,E), (m,F)$ with the edge subsets:
    \[ (i,j) \in ( E\otimes F )
    \Leftrightarrow ((\omega^{-1}_{n,m})(i)_1,(\omega^{-1}_{n,m})(j)_1) \in E
    \wedge ((\omega^{-1}_{n,m})(i)_2,(\omega^{-1}_{n,m})(j)_2) \in F, \]
    with $(x,y)_1 = x$ and $(x,y)_2 = y$ in this equivalence.

    Hence for two graph morphisms $\varphi\colon (n,E)\rightarrow (n',E'), \psi\colon (m,F)\rightarrow (m',F')$:
    \[ (\varphi \otimes \psi)(k) = \omega_{n',m'}( \varphi(\omega^{-1}_{n,m}(k)_1), \psi(\omega^{-1}_{n,m}(k)_2) ). \]
}

\lemma{
    For each pair $n,m\in \N$ we have natural bijections:
    \[ \sigma_{n,m}\colon \{0,\ldots, n+m-1\} \rightarrow \{0,\ldots, m+n-1 \}\]
    and
    \[ \theta_{n,m}\colon \{0,\ldots, nm-1\} \rightarrow \{0,\ldots, mn-1 \}.\]
    They are given by:
    \[
    \sigma_{n,m}(k)\colon=
    \begin{cases}
    k + m     & 0 \leq k \leq n-1 \Leftrightarrow m \leq k+m \leq n+m-1 \\
    k - n     & 0 \leq k-n \leq m-1 \Leftrightarrow n \leq k \leq n+m-1
    \end{cases}
    \]
    and:
    \[
    \theta_{n,m}(k)\colon=
       \floor{ k / n } + m \cdot ( k \mathrm{mod} n ).
    \]
    Both satisfy symmetry and associativity relations as follows:
    Symmetries
    \[
    \begin{aligned}
        \sigma_{n,m}\circ\sigma_{m,n} &= id_{\{0,\ldots,m+n-1\}} \\
        \sigma_{m,n}\circ\sigma_{n,m} &= id_{\{0,\ldots,n+m-1\}} \\
        \theta_{n,m}\circ\theta_{m,n} &= id_{\{0,\ldots,mn-1\}} \\
        \theta_{m,n}\circ\theta_{n,m} &= id_{\{0,\ldots,nm-1\}} \\
    \end{aligned}
    \]
    Associativity % es sollte egal sein, wie ich von lmn zu nml komme, so auch mit summe
    \[
    \begin{aligned}
        \sigma_{n,m}\circ\sigma_{m,n} &= id_{\{0,\ldots,m+n-1\}} \\
        \sigma_{m,n}\circ\sigma_{n,m} &= id_{\{0,\ldots,n+m-1\}} \\
        \theta_{n,m}\circ\theta_{m,n} &= id_{\{0,\ldots,mn-1\}} \\
        \theta_{m,n}\circ\theta_{n,m} &= id_{\{0,\ldots,nm-1\}} \\
    \end{aligned}
    \]
}

\thm{
    The category $\dag$ is bipermutative with $\oplus$ the disjoint union and symmetry isomorphisms $\sigma_{\bullet,\bullet}$, and $\otimes$ the product with symmetry $\theta_{\bullet,\bullet}$.
    The category has strictly sum-neutral element the empty dag and strictly product-neutral element the one-point dag.

    The category has equalisers and coequalisers for each pair of morphisms, in particular finite pullbacks and pushouts.

    There are combinatorial functors modelling cones $C$, as well as suspensions $\Sigma$.
    \begin{proof}
    \end{proof}
}

\thm{
    There is a functor $Fl\colon \dag \rightarrow sSet$, which assigns to each dag $G=(V,E)$ as $n$-simplices
    the graph homomorphisms from $\Delta^n$ to $G$. Faces and degeneracies are induced by the relevant
    maps on $\Delta^n$.

    The functor is strictly naturally sum preserving, that is $Fl(G\sqcup H) = Fl(G)\sqcup Fl(H)$ on objects
    as well as morphisms, also $F(\emptyset) = \emptyset$.

    The functor is strictly naturally product preserving, that is $Fl(G\times H) = Fl(G)\times Fl(H)$ on objects
    as well as morphisms, also $F(*) = *$.

    In addition the functor satisfies: \[ ( Fl\circ C ) = ( C \circ Fl ) \]
    for an appropriately chosen cone-functor $C$ on simplicial sets,
    as well as \[ ( Fl\circ \Sigma ) = ( \Sigma \circ Fl ) \]
    for a similar suspension-functor $\Sigma$ on simplicial sets.
    \begin{proof}
    By the universal property of the product $\otimes$ on $\dag$ we get on $n$-simplices:
    \[ \dag(\Delta^n,G\times H) = \dag(\Delta^n,G) \times \dag(\Delta^n,H), \]
    which exactly fits the description of the categorical product on simplicial sets.
    For sums, since $\Delta^n$ is a connected graph, on $\dag$ we get on $n$-simplices:
    \[ \dag(\Delta^n,G\sqcup H) = \dag(\Delta^n,G) \sqcup \dag(\Delta^n,H), \]
    which does not preclude that there might be more splittings of $G$ or $H$, but it does prove
    the compatability with disjoint union of both categories.
    \end{proof}
}

\chapter{Combinatorial Homotopy of DAGs}
\section{Homotopic Maps}
\Remi{
    Recall we have defined $I = C\S^0$, giving the dag:
    \xymatrix{
        0 & 2 \ar[l] \ar[r] & 1 .
    }
}

\chapter{Restart DAGs 2024-03-05}
\section{Core Definition: The small Category of finite directed acyclic graphs.}
\defn[Category of DAGs]{
    In this text THE category of finite directed acyclic graphs $\DAG$ is defined as follows.

    The objects are finite subsets of non-decreasing pairs of natural numbers:
    \[ \DAG_0 = \{~E\subset \N\times\N ~|~ \forall (x,y) \in E \colon x\leq y \wedge (x,x) \in E \wedge (y,y) \in E ~\}. \]

    Say the edge set $E$ describes a directed graph $G(E)=(V,E)$. Call the edges of type $(x,x)$ vertices, i.e. "the vertex $x$". Since the
    set $E$ is finite, there is a maximal $n\in\N$ appearing in all the pairs of $E$, call that the height $height(E)$. Also there is a
    finite amount of vertices, i.e. edges $(x,x)$, in that edge set, call the number of vertices the degree $deg(E)$.

    Morphisms are those maps of natural numbers that are monotonic with respect to the natural total order on natural numbers and respect these subsets:
    \[ \DAG_1(G_1,G_2) =\{~ \varphi \in \Set(\N,\N) ~|~ \forall (e_0,e_1) \in G_1 \colon (\varphi e_0, \varphi e_1)\in G_2 ~\}, \]
    (modulo an equivalence relation saying we don't care what $\varphi$ is beyond $G_1$'s range.)

    In particular given our definition of objects, it is acceptable to collapse an edge $(x,y)$ onto a vertex with a map that
    satisfies $\varphi x = \varphi y$, but it is not acceptable to "break" an edge.
}

\rem{
    Do notice that this definition still massively overrepresents even each isomorphism type of directed acyclic graph. The initial object, the
    empty graph represented by the empty subset of $\N\times\N$, is fortunately unique in this representation. The terminal object, the one-point
    graph however can be represented by any one-point set $\{(n,n)\}\subset \N\times \N$, so there are countably many isomorphic copies of those.

    More generally it is quite clear that any edge set is isomorphic to an edge set with equal height and degree.

    However, it is evidently a small category, since all its objects and morphisms are explicitly subsets of $\N\times\N$ and $\Set(\N,\N)$ respectively.
    Hence the whole category has cardinality at most $\Set(\N,\N)=\aleph_1$. I shall make an effort to keep everything this small to keep
    it accessible to machine computation.
}

\ex{
    The complete graph on $n$ nodes or equivalently the $n$ simplex in a simplicial sense $\Delta^n$ can be represented as the object:
    \[ \Delta^n = \{~ (i,j) \in \N\times\N ~|~0\leq i\leq j\leq n ~\}. \]
    In particular each graph whose maximal vertex is $n$ can be embedded in such a $\Delta^n$, even uniquely given our model choices.

    The simplicially experienced would expect me to write down $\partial\Delta^n$ now, this is a bit more effort in this category with very stiff
    objects and morphisms, from $n\geq 2$ on one needs one subdivision step to represent $\partial\Delta^n$ as a dag.

    However, we can conveniently describe the horns:
    \[ \Lambda^n_k = \{~(i,j) \in \N\times\N~|~0\leq k\leq n \wedge ( (0\leq i\leq j \leq k) \vee (k\leq i \leq j \leq n))~\}, \]
    which have the evident inclusions:
    \[ j\colon \Lambda^n_k \rightarrow \Delta^n, \]
    given by the identity on vertices, and since the horn condition is stronger than the complete graph condition, this is a dag map.
}

\section{The Core Construction: Subdivision}
TODO: hier ist harter off-by-one trouble drin .. fuegt Sd n basispunkt hinzu? sollte ich alles punktieren?
\lemma[The Power Set Map]{
    For $P_{fin}(\N) = \{~ S\subset \N ~|~ |S|<\infty~\}$ the finite subsets of natural numbers we can define a map:
    \[ b\colon P_{fin}(\N) \rightarrow \N \]
    by assigning to a finite set its binary encoding:
    \[ \{n_0,n_1,\ldots,n_k\} \mapsto \sum_{0\leq i \leq k} 2^{n_i} ( - 1 ? ). \]
    It is evidently injective, one can recover each $n_i$ as the binary digit indices where there is a $1$. It is also surjective, each
    natural number can be written as a finite binary word. Call the inverse $\pi\colon \N \rightarrow P_{fin}(\N)$, after all its target
    is the power set of its domain :) So for each $n\in \N$ we have a unique corresponding finite subset $\pi(n) \in P_{fin}(\N)$.
    The map has another very pleasant property, it is monotonic. Specifically, a proper subset inclusion corresponds to adding
    $1$s in the binary representation, so the number increases.
}

\defn[Subdivision of DAGs]{
    Let $E(G) \subset \N \times \N$ describe a directed graph $G$.

    Define the edge set
    \[E(Sd(G)) = \{~ (n,m) =(\sum_{i} 2^{s_i}, \sum_{j} 2^{t_j})  ~|~  \] \[ \{s_i\}_i\subset\{t_j\}_j \forall i,j\colon (s_i,t_j) \in E(G) \wedge (s_i,s_j)\in E(G) \wedge (t_i,t_j)\in E(G) ~\}.\]
    In words: The vertices are the numbers $n = \sum_i 2^{s_i}$ such that each non-decreasing pair $(s_i,s_j)$ is an edge in $E(G)$, call those legal vertices.
    Between each such pair of legal vertices $n=\sum_i 2^{s_i},m=\sum_{j} 2^{t_j})$, set $S=\{s_i\}_i,T=\{t_j\}_j$. Add an edge $(n,m)$ if and only if $S\subset T$.
}

\rem{
    Do note that since $S\subset T$ is an inclusion of legal vertices, each "new" edge has already been "tested" by $T$.
}

\ex[Subdivision of $\Delta^n$]{
    Recall $\Delta^n$ is represented by the edge set $\{(i,j)|0\leq i \leq j \leq n\}$.
}

TODO: erlaubte quotienten ... $\Delta^n$ zweimal subdividen und dann rand weg ist erlaubt und gibt $\S^n$.

TODO: schwache aequivalenzen sind hin und rueck kompositionen von acyclic cofibrations und acyclic fibrations.. also blow-up und collapse

\section{Cofibrations, Fibrations, Weak Equivalences, Homotopy, Connected Components, Factorisations, Cell Structures}

Cofibrations, alles injektive .. (cw struktur wiederfinden mit gutem path filtration argument)

w.eq's, abb'n $f\colon X\rightarrow Y$ such that $\exists g\colon U \rightarrow V$ and acyclic cofibrations (circular, vorsicht)



\end{document}
