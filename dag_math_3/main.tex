% This file is part of beamerthemepureminimalistic.

% If problems/bugs are found or enhancements are desired, please contact
% me over: https://github.com/kai-tub/latex-beamer-pure-minimalistic

\documentclass[a4paper]{book}
\include{version}
\usepackage{amsmath,amssymb,amsthm}
\newtheorem{thm}{Theorem}[section]
\newtheorem{theorem}{Theorem}[section]
\newtheorem{lem}[thm]{Lemma}
\newtheorem{lemma}[thm]{Lemma}
\newtheorem{prop}[thm]{Proposition}
\newtheorem{proposition}[thm]{Proposition}
\newtheorem{cor}[thm]{Corollary}
\newtheorem{corollary}[thm]{Corollary}
\theoremstyle{definition}
\newtheorem{Remi}[thm]{Reminder}
\newtheorem{rem}[thm]{Remark}
\newtheorem{defn}[thm]{Definition}
\newtheorem{ex}[thm]{Example}
\newtheorem{nonex}[thm]{{\bf NON}-Example}
\newtheorem{conj}[thm]{Conjecture}
\newtheorem{Exercise}[thm]{Exercise}
\newtheorem{Code}[thm]{Pseudocode}
\usepackage{datetime2}
\usepackage{graphicx}
\usepackage{svg}
% \svgpath{{/home/vassago/LATEX/}{/home/vassago/LATEX/svgs/}}
\usepackage{svg-extract}
\usepackage{hyperref}
\usepackage{inconsolata}

\usepackage{tcolorbox}

\usepackage[all]{xy}
\usepackage{xypic}
\usepackage{xcolor}

\definecolor{light}{RGB}{0,100,43}
\definecolor{dark}{RGB}{130,0,100}
\definecolor{exercise}{RGB}{220,200,255}
\definecolor{textcolor}{RGB}{0, 150, 128}
\definecolor{title}{RGB}{150, 0, 128}
\definecolor{footercolor}{RGB}{0, 128, 64}
\definecolor{code}{RGB}{240,220,160}
\definecolor{codebg}{RGB}{50,50,50}
\definecolor{mathbg}{RGB}{0,10,30}
\definecolor{mathtext}{RGB}{220,220,250}
\definecolor{bg}{RGB}{32, 0, 16}

\def\exercise#1{\color{light}{\h\Exercise{\color{exercise}\scshape #1}}\color{textcolor}}
\def\code#1#2{\color{dark}{\h\Code{\h}\\[.5em]%
\color{mathtext}#1
\begin{tcolorbox}[colback=codebg]%
\color{code}%
\texttt{#2}%
\color{textcolor}%
\end{tcolorbox}\color{textcolor}}}

\def\backtick{`}
\def\curtime{\DTMdate{2023-09-21} ~\DTMtime{01:42:17}\DTMdisplayzone{-5}{00}}
\def\light#1{{\color{light}#1}}
\def\dark#1{{\color{dark}#1}}
\def\h{\hspace{1em}}

\def\signature{\href{mailto:marc@lange-data.org}{Dr. Marc Lange, marc@lange-data.org}}
\def\mailsignature{\href{mailto:marc@lange-data.org}{marc@lange-data.org}}

\def\abs#1{\left| #1 \right| }
\def\floor#1{\lfloor #1 \rfloor }

\newcommand{\lthree}{<\hspace{-5.2pt}3}
\newcommand{\E}{\mathcal{E}}
\newcommand{\DAG}{\mathcal{DAG}}
\newcommand{\Z}{\mathbb{Z}}
\newcommand{\R}{\mathbb{R}}
\renewcommand{\P}{\mathfrak{P}}
\newcommand{\Q}{\mathbb{Q}}
\newcommand{\N}{\mathbb{N}}
\renewcommand{\S}{\mathbb{S}}
\newcommand{\C}{\mathcal{C} }
\newcommand{\D}{\mathbb{D}}
\newcommand{\LL}{\mathcal{L}^1(\Z,\Z)}
\newcommand{\Set}{\mathrm{Set}}

\renewcommand{\dag}{\mathrm{emb}\mathcal{DAG}}


\newcommand{\s}{Set^{\phantom{\leq}}_{\phantom{+}}}
\newcommand{\spp}{Set=fSet}
\newcommand{\smp}{Set^{\leq}_{\phantom{+}}}
\newcommand{\spm}{Set^{\phantom{\leq}}_{+}}
\newcommand{\smm}{Set^{\leq}_{+}}
\newcommand{\spmpm}{Set^{\pm}_{\pm}}

\newcommand{\App}{A}
\newcommand{\Amp}{A^{\leq}_{\phantom{+}}}
\newcommand{\Apm}{A^{\phantom{\leq}}_{+}}
\newcommand{\Amm}{A^{\leq}_{+}}

\begin{document}
\chapter{2024-03-20 -- Sets, Relations, Transitivity, Partial and Total Orders}
\section{The fundamental set categories}
This paper makes liberal use of the language of ($1$-)categories, specifically adjunctions are
numerous in these results. The unfamiliar reader is encouraged to look up their favourite adjunction
anywhere, I can guarantee you have one, if you do not know it. Following along the adjunctions here
inspired by any other adjunction, for example of the free-forgetful-type, helps clarify the roles of
source- and target-categories in each adjunction here. For a more rigorous reference Saunders MacLane's
"Categories for the Working Mathematician" will do nicely, there is no ultra-modern category tech here.
However, for definiteness fix what categories are in this paper:
\defn{
    Consider a tuple $\mathcal{C}=(\C_0,\C_1,s,t,id,\circ)$ consisting of two sets $\C_0,\C_1$ in the sense recalled in the next definition
    and four set-maps $id\colon \C_0 \rightarrow \C_1, s,t\colon \C_1\rightarrow \C_0, \circ \C_1\times_{t,s}\C_1\rightarrow \C_1,$
    where $\C_1\times_{t,s}\C_1 := \{~(c,d)\in\C_1\times\C_1 ~|~t(c)=s(d)~\}$.

    Call the above structure a $1$-category, if the following compatibility conditions are satisfied:
    \[ \circ(\circ(f,g),h) = \circ(f,\circ(g,h)), \circ(1_\bullet,f)=f, \circ(f,1_\bullet)=f.  \]

    No person would write composition like that though, so introduce the notation $d\circ c := \circ(c,d)$.
    Then the axioms become the more familiar:
    \[ (f\circ g)\circ h = f\circ(g\circ h), 1_\bullet\circ f = f, f\circ 1_\bullet=f. \]
}

\rem{
    Note that in particular from the perspective of this paper all categories are small, as in, the set of
    all objects and all morphisms is actually a set. Obviously the category of all sets or all relations do not
    satisfy having sets of objects like above, but still satisfy the relations. The distinction between small
    and large categories in this paper is however an easily noticeable contrast even without emphasis.
}

\rem{
    In light of the associativity of $\circ$ one could make an argument to denote $n$-fold composites as $\circ(f_n,\ldots,f_1)$ though.
}



Since a lot of the following arguments rely on very specific descriptions of the involved sets, it
seems wise to recall a few of the set-theoretic notions here, also to fix notation. They are however classical,
and strictly speaking a lot of the results of this paper can be recovered in far smaller models of set.
It, however, seems opportune to state in proper generality for all sets all that can be done for such
general sets.
\defn{
    A set for the purposes of this paper is a collection of elements $S = \{s|s\in S\}$ which uniquely
    identify the set, i.e. given another set $T = \{t|t\in S\}$, both are equal if and only if they have
    the same elements:
    \[ S=T \Leftrightarrow \forall s \in S \colon s\in T \wedge \forall t \in T \colon t \in S. \]
    A map of sets $f\colon S\rightarrow T$ is informally a "formula" that for each element of
    $S$ unambiguously assigns an element of $T$. For example the set of all maps between $S,T$ sets can
    be realised as a set explicitly as follows:
    \[ Set(S,T):=\{~f\subset S\times T~|~\forall s\in S \exists! t\in T\colon (s,t)\in f~\} .\]
    Maps are the binary relations between the sets $S$ and $T$ such that each source element has a
    unique target element, not necessarily different for different $s\in S$. For $f$ an element
    of $Set(S,T)$ hence write the guaranteed unique element of $T$ that is in $f$-relation to it as its
    $f$-image, i.e. $f(s)=t :\Leftrightarrow (s,t)\in f$.
    Maps are hence equal if and only if they are equal as sets if and only if
    they agree on each source element. Maps can be composed in the familiar way, identities are
    realised as the diagonal for each set. The fact that maps compose associatively is a well-known fact
    of set theory just quoted here.

    Call the category of sets thus "defined" $Set$.
}
\rem{
    Note that implicitly one has to fix an inaccesible cardinal here, and limit the cardinality of the included sets below that cardinal. Then one
    can accurately identify $Set$ as a hierarchy of $1$-categories of ever increasing largeness, where the definition of what is allowed to
    be considered a set extends constantly.

    For many applications that is an exceptionally important point, for these applications it is not, hence suppress such considerations wherever
    convenient here.
}
\rem{The axiomatically minded reader is encouraged to take this definition as, assume the ZFC axioms,
     and a model of category theory of "small" categories in it, i.e. categories that have object- and morphism-sets
     constructed by these ZFC axioms of our universe. All categories of interest in this paper are small (hence also locally small).}
\rem{Note that classically in set one can construct arbitrary unions, arbitrary disjoint unions, arbitrary products (by axiom of choice),
     subobjects, quotients, colimits, limits. Sets are a very convenient category in which to construct base objects, but usually not
     structured enough.}

\defn{
    A binary relation $R$ on two sets $S,T$ is a subset of their product
    $R\subset S\times T$, denote $s\in S, t\in T$ to be in
    $R$ relation as $sRt:\Leftrightarrow (s,t)\in R$.

    A map of relations $f\colon R_0\rightarrow R_1$ is a pair of maps $f_S\colon S_0 \rightarrow S_1$ and $f_T\colon T_0\rightarrow T_1$ for
    $R_i\subset S_i\times T_i$, such that restricting $f$ to elements of $R_0$ only yields elements of $R_1$, say the morphism respects
    the relation: \[ f~ \mathrm{relation ~morphism}~:\Leftrightarrow~\forall (s,t) \in R_0\colon (f_S(s),f_T(t))\in R_1. \]
}
\rem{Since composition of heteregoneous relations restricts to composition of set maps, recall the definition for clarity.}
\defn{
    Given two binary relations $ R_0\subset S\times T, R_1\subset T\times U$ with compatible middle set $T$, define the
    composite relation $R_1\circ R_0 \subset S\times U$ as follows:
    \[ (s,u) \in R_1\circ R_0 :\Leftrightarrow \exists t \in T\colon sR_0t \wedge tR_1u. \]

    I.e. the composite relation contains all such pairs such that in the middle step there exists a common relating element
    in the "forgotten set" $T$.
}

\rem{The only such general relations that feature in this content are the set maps defined as above. Note that composition of relation is just as
     associative as composition of set maps.

     It is convenient for this paper to restrict to a more restrictive class of relations.}

\defn{
    A homogeneous binary relation $R$ on a set $S$ is a subset of the product of $S$ with itself $R\subset S\times S$.
    Call $S$ the underlying set of $R$.

    A map of homogenous binary relations $f\colon R_0\rightarrow R_1$ is a map of the underlying sets $f\colon S\rightarrow T$ that
    respects the relation, i.e. $\forall s,t \in S: sR_0t \Rightarrow f(s)R_1f(t)$. Maps contain identities, they compose, and associatively, because
    that is true on $Set$.

    Call the category of homogeneous binary relations thus defined $Rel$.

    Note in addition that composition of relations restricts to homogeneous relations.
}

\rem{
    Do note that in particular each map of underlying sets induces a map of the empty relations.
}

\prop{
    Consider for each non-empty set the empty relation as follows $R=(\emptyset\subset S\times S)$, then for any other
    non empty relation there are not any maps into $R$. For any other relation, empty or not the set of maps from the
    empty relation to it is the same as a map of the underlying sets. In particular the relation $\emptyset$ on the
    underlying set $\emptyset$ is an initial object of the category of relations.
    \begin{proof}
    If the relation is non-empty, there is no way to define a map from it into an empty set, in particular the
    underlying set map cannot induce a compatible map on relations. Hence follows the first claim.

    By definition of relation-maps such a map is a map of underlying sets, respecting relations. Since there are not any
    relations to respect in the empty relation, such a map is always trivially a relation-map, so all set maps
    induce a relation-map.
    \end{proof}
}

\prop{
    The one-point set with relation equality $(\{*\},\{(*,*)\})$ is a terminal object in $Rel$.
    \begin{proof}
    As a corollary to the above proposition, the largest a set of relation morphisms can get, is, if the source relation
    is empty. In that case the map and the fact that it is a map of relations is equivalent to the underlying set map.
    For that case there exists only one underlying set map to any one-point set. Hence, since the terminal object displayed
    above has all relations between its elements, any set map to the one-point set defines a unique relation-map. Hence
    the resulting relation-map is unique.
    \end{proof}
}

\defn{
    Every set $S$ has two trivially associated homogeneous relations, the discrete one and the collapsible one. Fix
    notations in accordance with the upcoming topological interpretations:
    \[ S^\delta = \{~(s,s)~|~s\in S~\} ~~~ \Delta^S = \{~(s,t)~|~s,t\in S~\}. \]
    In addition note that every relation on $S$ has a unique morphism from $S^\delta$, which is the identity on $S$, and a unique
    morphism to $\Delta^S$, which is the identity on $S$.
}

\defn{
    Call a relation with any underlying set an empty relation if it has no elements.
    Call a relation discrete if it can be realised as $S^\delta$ by its underlying set. % == reflexive closure of empty relation!
    Call a relation collapsible if it can be reliased as $\Delta^S$ by its underlying set.
}

\rem{
    Note that interpreted in the context of sets and set maps the discrete relation is exactly the identity map of its underlying set.
    Collapsible relations are never maps for sets with at least two elements, but the empty set and one point set each induce the
    identity on themselves this way too.
}

\lem{
    Morphisms into collapsible relations are the same as the set maps of underlying sets, i.e. for $U$ the forgetful
    functor as above, there is an adjunction:
    \[ U\colon Rel \leftrightarrow Set \colon \Delta^\bullet. \]
    With $U$ the forgetful functor $Rel\rightarrow Set$, synonymously with "underlying set functor", as above.
    \begin{proof}
    The claim is $Rel(R\subset S\times S,\Delta^T)=Set(UR=S,T)$. Since $\Delta^T$ contains every possible pair, being a
    map of relations into it is an empty condition, so forgetting down to underlying sets retains all information and the
    argument works %todo!
    \end{proof}
}

\cor{
    Since colimits are defined as left adjoints follows by the standard $1$-category argument, colimits of relations
    are created by forgetting down to sets. I.e. take the colimit of the relations and their underlying sets componentwise in $Set$,
    that is a valid colimit in $Rel$.
}

\lem{
    There is an isomorphism of categories: Sets and set maps are the same thing as the full subcategory of binary homogeneous discrete relations.
    \begin{proof}
    Any set can be represented as its associated discrete relation, any map of sets trivially induces a map of discrete relations, because the
    condition of being a relation map is vacuously satisfied. Hence forgetting down the discrete relation recovers the same sets and maps.

    Coming from a discrete relation, forgetting down to the underlying set forgets nothing about the maps, because they do not have to satisfy non-trivial
    relation-compatibility conditions on discrete relations. Hence they are uniquely described on their underlying sets and each such map induces
    a map of relations by passing to the discrete relations again, recovering the relations and their maps.
    \end{proof}
}

\rem{
    It might seem brutal notation to denote by $Rel$ such a stricter category of relations than the more general definition above.
    However, apart from maps the relations in this paper that structure the content are all homogeneous, so they deserve emphasis.
}

%\section{Endofunctors of $Rel$, Transitivity, Partial Orders, Total Orders}
\section{Reflexive Relations are complete and cocomplete}
The category of (homogeneous binary) relations $Rel$ a few interesting endofunctors, consider the first one of this paper.
\defn{
    For each relation $R\subset S\times S$ consider its reflexive closure:
    \[ refl\colon Rel\rightarrow Rel \]
    with $refl(R):=\{~(s,s)~|~s\in S~\} \cup R$. Its naturally compatible
    with maps of (homogeneous) relations, hence a functor.

    Call a relation $R$ reflexive if it satisfies $relf(R)=R$, which is exactly the case if all diagonal elements
    of the underlying set were already part of $R$.

    Call the full subcategory of reflexive relations $RRel$, i.e. $RRel$ has objects relations that are $refl$-fixed points
    and all relation morphisms between them.
}

\rem{
    One can equivalently, since the subcategory is full, consider the reflexive closure the left-adjoint to
    the forgetful functor from reflexive relations to relations:
    \[ refl\colon Rel \leftrightarrow RRel \colon U. \]
    That point does not clarify things substantially here, but might be helpful in other contexts.
}

\lem{
    Morphisms out of a discrete relation into any reflexive relation are the same as the set maps of the underlying sets. I.e.
    send each reflexive relation $RRel\ni R\subset S\times S$ to its underlying set $UR = S\in Set$, and similarly on maps, then
    there is an adjunction:
    \[ (\bullet)^\delta\colon Set \leftrightarrow RRel \colon U. \]
    Call $U$ the forgetful functor $RRel\rightarrow Set$, synonymously with "underlying set functor".
    \begin{proof}
    Staring at $RRel(M^\delta,R\subset S\times S) = Set(M,S)$ long enough will convince the reader this is true.
    Explicitly: A map of homogeneous relations from a discrete relation only is a map of equal pairs $(m,m)$, hence
    these are equal in $S$ too, so forget down to the underlying set map. Since a morphism of relations is defined
    via the underlying set map with properties, which are vacuously satisfied for a discrete source object follows
    the above equality. Note however that only reflexive $R$'s admit such maps in this way, thus the restriction to that
    subcategory.
    \end{proof}
}

\cor{
    The adjunction as above restricts to the subcategory of reflexive relations.
    \[ U\colon RRel \leftrightarrow Set \colon \Delta^\bullet. \]
    \begin{proof}
    The claim is $Set(UR,S) = RRel(R,\Delta^S)$, i.e. for a reflexive relation $R$ the relation morphisms to a collapsible relation
    with underlying set $S$ are exactly the same as set morphisms from $R$'s underlying set. Since $RRel$ is a full subcategory
    of $Rel$ this follows from the analogous lemma on $Rel$, since $\Delta^S$ is a reflexive relation for every set $S$, hence also
    contained in $RRel$.
    \end{proof}
}

\rem{Note interestingly that making a relation irreflexive is not a functor without assuming a smaller class of maps. If $f\colon R_0 \rightarrow R_1$
being a map of relations depended on a relation $xR_0y$ with $x\neq y$ and $fx=fy$ with $fxR_1fy$, then making $R_1$ irreflexive breaks the
property of $f$ being a morphism of relations. So any non-injective map of homogeneous relations does not induce a map on the associated irreflexive relations.}

\rem{
    The adjunctions establish an interesting conflict of intuitions. There should clearly be more relations than there are plain sets,
    which is expressed by the fact that $\Delta^S$ allow for any kind of map into them, so recover all set maps of the underlying sets of all relations,
    but there are clearly more relations that are not $\Delta^S$ for any non-empty set $S$.

    On the other hand the other adjunction involving the discrete relations $(\bullet)^\delta$ is a formalisation of the fact that
    a map of relations is uniquely determined by what it does on the underlying sets, everything else are conditions. Hence because
    every relation admits a map from the discrete relation on its vertex set with underlying set map the identity, see in particular
    that a map of relations is uniquely determined on the underlying set, and it follows from the adjunction.
}

\cor{
    By identifying sets with the category of discrete relations the above adjunctions can actually be interpreted as adjunctions to an
    inclusion $i\colon Set \hookrightarrow RRel$.
    \begin{proof} This is basically a matter of notational preference, but important enough to emphasise. \end{proof}
}

Finally note the result for which the adjunctions were set up:
\thm{
    The category of reflexive relations is complete and cocomplete, with colimits and limits computed componentwise for
    objects as well as morphisms.
    \begin{proof}
    The plain existence of left- and right-adjoints to the forgetful functor to sets guarantees the above fact.

    Consider for simplicity and explicity inductive limits and colimits:
    Given a system $(R_0\rightarrow R_1\rightarrow \ldots\rightarrow R_n\rightarrow R_{n+1}\rightarrow \ldots)$ of arbitrary maps
    of reflexive relations and their underlying sets, define the colimit underlying set as the colimit of the underlying sets, then the
    induced relation is defined exactly by being related in a finite step. This satisfies the universal property of the colimit
    in reflexive relations, hence is a reflexive relation representing that colimit.

    Similarly given a system $(\ldots \rightarrow R_{n+1}\rightarrow R_n\rightarrow \ldots \rightarrow R_1\rightarrow R_0)$ of arbitrary
    maps of reflexive relations and underlying sets, define the limit underlying set as the limit of underlying sets. Elements are in relation if they
    are so in each $R_i$ step. On underlying sets the universal property is inherited from sets, and the extended relation is uniquely forced
    by that structure.

    Arbitrary diagrams follow similarly in both cases, simply replace "in a finite step" with "in some representing summand for the quotient" in the colimit,
    the limit-case works verbatim. The limit and colimit relations are each well-defined because the system-maps are maps of relations.
    \end{proof}
}

Each of the following corollaries follows directly from the theorem, but it is worth explicitly constructing some special cases of limits and colimits,
which are of further use here.

\cor{
    Reflexive Relations have pushouts.
    \begin{proof}
    Given three reflexive relations $(R_0\subset S_0\times S_0,R_1\subset S_1\times S_1,R_2\subset S_2\times S_2)$, with set maps $i_1\colon S_0\rightarrow S_1$ and
    $i_2\colon S_0 \rightarrow S_2$ each inducing relation maps on the $R_i$. Then consider the pushout of underlying sets $S_1\cup_{S_0}S_2$ with
    set maps $j_1\colon S_1\rightarrow S_1\cup_{S_0}S_2$, $j_2\colon S_2\rightarrow S_1\cup_{S_0}S_2$.

    By construction of colimits in set an element $s\in S_1\cup_{S_0}S_2$ can be represented by either elements of $S_1$ or $S_2$, with identifications over $S_0$.
    Hence set the quotient relation to be $\lbrack x \rbrack R \lbrack y \rbrack :\Leftrightarrow (x,y\in S_1, xR_1y) \vee (x,y\in S_2,xR_2y)$. It is clearly a relation,
    since it has pairs in the same underlying set, and it is reflexive, because all relations involved are.

    It satisfies the universal property of the pushout, since it has the minimal set of relation pairs given the diagram, and the morphism from the common source $R_0$ ensures
    that each compatible pair of maps factors over the quotient.
    \end{proof}
}

\rem{
    Be aware that pushouts of relations are a relatively brutal construction, consider the following few examples.

    Let the relations $R_0,R_1,R_2$ be as follows: All are defined on the underlying set of two elements, denote it
    $S = \{0,1\}$, the maps are the ones induced by the identity of $S$.

    Let $R_0 = S^\delta$, and set $R_1 = S^\delta \cup \{(0,1)\}$ and $R_2 = S^\delta \cup \{(1,0)\}$.
    Then the pushout relation is defined on the same set $S$, contains the diagonal trivially, but also every
    possible pair, because $(0,1)$ and $(1,0)$ are introduced each from their summands. So pushing out the two
    relations that could be called $\leq$ and $\geq$ respectively, get the trivial collapsible relation on the underlying sets.
    This example generalises to arbitrary sets with at least two points.

    Consider also explicitly quotients, that is any map of relations $R_0\rightarrow R_1$ on arbitrary underlying sets $S_0\rightarrow S_1$,
    and consider the unique map $R_0 \rightarrow pt$ to the terminal relation, the one-point-relation on the one-point set.

    Then the resulting pushout has underlying set $S_1 / im S_0$, and the relation recovers exactly the relations on
    $S_1 \setminus im S_0$ while the whole underlying set $S_0$ has been collapsed to a point in the quotient, which satisfies reflexivity.
}

\cor{
    Reflexive relations have coequalisers.
    \begin{proof}
    Let $f,g\colon R_0 \rightarrow R_1$ be two morphisms of reflexive relations with underlying sets $S_0,S_1$ respectively.
    Consider the underlying quotient set $S = S_1/\sim$ with equivalence relation generated by $fx\sim gx \forall x \in S_0$.
    Induce a relation $R/\sim$ by the relation of $S_1$, i.e. $\lbrack x\rbrack R/\sim \lbrack y\rbrack :\Leftrightarrow
    \exists a,b\in S_1: a\in\lbrack x\rbrack, b\in \lbrack y\rbrack, aR_1b$. Since $R_1$ is reflexive, so is $R/\sim$. The universal property
    is readily verified.
    \end{proof}
}

\cor{
    Reflexive relations have pullbacks.
    \begin{proof}
    Let $f\colon R_1\rightarrow R_0$ and $g\colon R_2\rightarrow R_0$ be two morphisms of relations with a common target and underlying sets $S_i$ as before.
    Consider the pullback set $S = S_1\times_{S_0}S_2=\{~(s,t)\in S_1\times S_2~|~fs=gt~\}$ and induce a relation $R$ on $S$ componentwise:
    $(s,t)R(u,v) :\Leftrightarrow sR_1u \wedge tR_2v$. The universal property is easy to prove.
    \end{proof}
}

\cor{
    Reflexive relations have equalisers.
    \begin{proof}
    Let $f,g\colon R_0\rightarrow R_1$ be two parallel maps of relations on underlying sets $S_i$, and consider the set $S:=\{~s\in S_0~|~fs=gs~\}$, and
    induce a relation by restricting $R_0$ to that subset. It is a again a reflexive relation and clearly the pair satisfies the universal property
    of an equaliser of $f$ and $g$.
    \end{proof}
}

\section{Transitivity}
So far our basic categories include $RRel, Rel, Set$ connected by a few free-forgetful adjunctions.
Add another such category to the chain by considering transitive relations, for that introduce another
endofunctor on relations.
\defn{
    For an arbitrary relation $R \subset S\times S$ with underlying set $S$ consider the $i$-fold
    composition of $R$ with itself:
    \[ R^1 := R, R^i:= R\circ R^{i-1} \forall i>0. \]

    More explicitly $R^i = \{~(s,u)\in S\times S~|~\exists (t_1,\ldots,t_{i-1}) \forall j sRt_1, t_{i-1}Ru, t_jRt_{j+1} ~\}$ is
    the relation defined by all pairs that can be connected by a chain of relations of length $i-1$. Specifically the one-fold
    iterate is the relation itself with no intermediate connectors.

    Since the iterates are all subsets of a common superset $S\times S$ it also makes unambiguous sense
    to consider their union:
    \[ trs(R):= \bigcup_{i\geq 1} R^i, \]
    which then consists of the original relation $R$, as well as all pairs that were $R$-connectable by some finite
    sequence of consecutive $R$ relations of $S$ elements. Then call $trs(R)$ (with underlying set $S$) the transitive closure
    of $R$. It is a functor from relations to itself, call the full subcategory of transitive relations in all relations $TRel$,
    i.e. the category on objects relations $R$ such that $trs(R)=R$ with all relation maps between them.
}

There is a reflexive variation on this by introducing the zero-th exponent.

\defn{
    For an arbitrary relation $R \subset S\times S$ with underlying set $S$ consider the $i$-fold
    composition of $R$ with itself:
    \[ R^0 := \lbrack S \rbrack^\delta, R^i:= R\circ R^{i-1} \forall i>0, \]
    and define the unital transitive closure as the analogous union:
    \[ trs_1(R):=\bigcup_{i\geq0} R^i, \]
    it consists of all diagonal elements of the underlying set, hence $trs_1(R)$ is a reflexive relation
    for any relation $R$. Call the full subcategory of reflexive and transitive relations with all
    relation maps between them $TRRel$.
}

\lem{
    The transitive closure is both the left- and the right-adjoint in the free forgetful adjunction between the
    category of all relations and the category of transitive relations. It restricts to adjunctions on reflexive relations.
    \[ trs\colon Rel\Leftrightarrow TRel\colon U, \]
    \[ U\colon TRel\Leftrightarrow Rel\colon trs, \]
    and
    \[ trs\colon RRel\Leftrightarrow TRRel\colon U, \]
    \[ U\colon TRRel\Leftrightarrow RRel\colon trs. \]
    \begin{proof}
    The claim is $TRel(trs(R_0),R_1) = Rel(R_0,UR_1)$, as well as $Rel(U(R_0),R_1) = TRel(R_0,trs(R_1))$, and the according
    claim for reflexive relations. These are however all trivially satisfied, since the transitive and reflexive relation
    categories are defined as full subcategories of relations. In other words, there is nothing specifically reflexive
    or transitive a morphism of relations could satisfy or respect.
    \end{proof}
}

Hence follows as before for reflexive relations that the category of transitive reflexive relations has all limits and colimits.
\lem{
    The category of transitive reflexive relations is complete and cocomplete, the limits and colimits can be computed just
    as in relations, hence pointwise.
    \begin{proof}
    First consider limits, i.e. a candidate object $lim_I R_i$. The underlying set of $lim_I R_i$ clearly has to be the
    limit of the underlying $I$-diagram in set. When realised as subset of the product $\Pi_IS_i$, the induced relation is
    defined by two $I$-tuples in the $S_i$ being in $lim_I$ relation if and only if each of their components are. If each
    $R_i$ is reflexive, so is the limit-relation, if each $R_i$ is transitive, so is the limit relation. So pullbacks, equalisers,
    subobjects are all the underlying constructions of reflexive relations as above.

    Considering colimits is a bit more work because of the transitive closure involved.
    The most salient point about this adjunction is that reflexivity is inherited on quotients, but transitivity is not, so
    the transitive closure functor has to feature despite the apparent simplicity of the adjunction.
    Consider hence a candidate object $colim_I R_i$, with underlying set realised by $colim_I S_i$ as a quotient of the
    disjoint union of the $S_i$ modulo the equivalence relation generated by the $I$-diagram. On that quotient induce
    the relation \[\lbrack x\rbrack R \lbrack y \rbrack :\Leftrightarrow \exists i\in I, s,t \in S_i\colon \lbrack s\rbrack =\lbrack t\rbrack \in colim_I S_i \wedge sR_it.\]
    It is reflexive, because each contributing summand contributes the diagonal elements of its underlying set. However there
    is no reason for a colimit of transitive relations to be transitive itself and in fact it is not in proper generality.
    So take the transitive closure, which is the same as the unital closure on reflexive relations, $trs(colim R_i)=trs_1(colim R_i)$ of the
    colimit of the summand relations $R_i$, then this is evidently a transitive relation, which satisfies the universal property
    of the colimit in transitive relations.
    \end{proof}
}

\ex{
    Let us show an example where the pushout of two transitive relations would not be transitively closed without applying the closure functor.
    Consider the transitive and reflexive relations $ 1 < 3 < 5 $ and $2 < 3 < 4$ with pushout over $3$, also reflexive and transitive. Then
    the colimit relation $R$ has all these relations $R=\{(1,1),(3,3),(5,5),(2,2),(4,4),(1,3),(1,5),(3,5),(2,3),(3,4),(2,4)\}$. Hence notice,
    the first relation introduced a relation $1<3$ and the second one introduced a relation $3<4$, but on its own $R$ has no reason to also
    introduce $1<4$ in the colimit. Symmetrically in this example for $2<3$ and $3<5$, which needs an additional relation $2<5$.

    Specifically the minimal failing example is the pushout of two "edges" $1<2$ and $2<3$, because $1<3$ only follows after transitively closing.

    The example generalises into the following observation.
}

\lem{
    Note that for each set the discrete relation and the collapsible relation on it are reflexive and transitive relations.
    In the category of sets the following diagram is trivially a pushout for any two elements $s,t\in S$:
    \[
    \xymatrix{
    S\setminus \{s,t\} \ar[r]\ar[d] & S\setminus \{s\}\ar[d]\\
    S\setminus\{t\}\ar[r] & S.
    }
    \]
    Analogously the following diagram is a pushout in set for each two $s,t\in S$:
    \[
    \xymatrix{
    \{s\} \ar[r]\ar[d] & \{s,t\}\ar[d]\\
    S\setminus\{t\}\ar[r] & S.
    }
    \]

    These induce the following four pushout diagrams in transitive reflexive relations
    by taking discrete and collapsible relations:
    For the first diagram get:
    \[\xymatrix{
    (S\setminus \{s,t\})^\delta \ar[r]\ar[d] & (S\setminus \{s\})^\delta \ar[d]\\
    {(S\setminus \{t\})}^\delta \ar[r] & S^\delta,
    }\]
    and
    \[\xymatrix{
    \Delta^{S\setminus \{s,t\}} \ar[r]\ar[d] & \Delta^{S\setminus \{s\}}\ar[d]\\
    \Delta^{S\setminus \{t\}}\ar[r] & \Delta^S.
    }\]

    For the second diagram get:
    \[ \xymatrix{
    s^\delta \ar[r]\ar[d] & \{s,t\}^\delta \ar[d]\\
    {(S\setminus \{t\})}^\delta \ar[r] & S^\delta,
    } \]
    and
    \[ \xymatrix{
    \Delta^s \ar[r]\ar[d] & \Delta^{\{s,t\}}\ar[d]\\
    \Delta^{S\setminus \{t\}}\ar[r] & \Delta^S.
    } \]

    Each of these are pushout diagrams in the category of transitive, reflexive relations.
    \begin{proof}
    The universal properties are all readily verified since the underlying maps all involve the above
    mentioned pushout diagrams in sets, and the induced relations are identified as the above objects.

    The notable point is that without transitive closure the pushouts for $\Delta^\bullet$ would clearly
    not be transitively closed in general just like in the example above, hence the pushout in the
    category of reflexive and transitive relations of collapsible relations becomes another collapsible relation.
    \end{proof}
}
\section{Antisymmetry}
The upshot of this section is that antisymmetry is the only colimit/quotient-troublesome relation of an order relation. It cannot
be broken by limit constructions, but colimits of antisymmetric things do not need to be antisymmetric. In addition
there is no natural way to introduce a functor identifying antisymmetric relations as the image of that functor like
for reflexivity and transitivity. Essentially the fact that the condition induces for each non-trivial relation a non-trivial
non-existence condition can break antisymmetry of pushouts.

\defn{
    A relation $R\subset S\times S$ is called antisymmetric, if for all pairs of elements $s,t\in S$ it follows $sRt \wedge tRs\rightarrow s=t$.
    I.e. there is only ever at most a relation in one direction and not the other.

    Denote the full subcategory of all antisymmetric relations, i.e. on objects relations that are asymmetric and with morphisms all relation maps
    between them, as $ARel$. Denote the category of antisymmetric reflexive relations $ARRel$, the category of antisymmetric, transitive, reflexive
    relations $ARTRel$, or by the more common name partially ordered sets, or posets, $Poset$.
}

\ex{
    Consider the two relations $\{(0,0),(1,1),(0,1)\}$ and $\{(0,0),(1,1),(1,0)\}$ with intersection the discrete relation
    $\{(0,0),(1,1)\}$. All three involved relations are antisymmetric, i.e. $\forall x,y \in S\colon xRy \wedge yRx \Rightarrow x=y$,
    but in the pushout there is the relation $(0,1)$ and the relation $(1,0)$ while $0\neq1$, violating antisymmetry.
}

This naturally leads to two concepts on general relations.
The dual object of this is of more immediate use here, but the subobject concept seems easier to understand as the first definition.
\defn{
    For any relation $R\subset S\times S$ consider its maximal antisymmetric subrelation $R_<$ as follows, the underlying set $S_<$ is the
    subset of $S$, where $R$ restrict to an antisymmetric relation:
    \[ S_< = \{~s\in S~|~\forall t\in S\colon sRt\wedge tRs \rightarrow s=t  ~\}, \]
    and the induced relation is the restriction of $R$.
}
\prop{
    The maximal antisymmetric subrelation of any relation is in fact antisymmetric.
    \begin{proof}
    By definition of $S_<$ this follows for each pair from the condition in two redundant ways.
    \end{proof}
}
\ex{
    Notice that the maximal antisymmetric subrelation is maximal in that it allows a natural description for arbitrary relations,
    there may well be antisymmetric relations properly between the whole relation and the maximal antisymmetric quotient as defined above, which are
    obtained by restricting the relation too, not just the underlying set. There seems to be no natural way to antisymmetrise a relation by
    picking a subrelation without additional structure on the underlying sets like a total order. Presumably the argument would have to
    involve strictifying the axiom of choice in the guise of the well-ordering theorem into choosing a well-ordering for each set in a way
    that becomes a functor after everything is picked. Those details are beyond the scope of this paper in this generality.
}
\rem{
    The maximal antisymmetric suboject is not needed for the constructions of limits, in particular equalisers, subobjects, etc., since the reflexive relation limits
    of an $I$-diagram with only antisymmetric relations and relation morphisms between them, are antisymmetric without any additional assumptions or constructions.
    Phrased differently, the maximal antisymmetric subobject of an antisymmetric relation is obviously the relation itself, hence limits of antisymmetric relations
    agree with their maximal antisymmetric subobjects too.
}
\prop{
    The (reflexive) relation limits of antisymmetric relations are antisymmetric.
    \begin{proof}
    In fact reflexivity is not needed for the limit to exist.
    Note that the induced relation is defined as being in relation in each component. Hence follows from a relation $xRy$ and $yRx$ in the limit
    a relation in each component, so $x$ and $y$ are equal in each component, hence $x=y$.
    \end{proof}
}
\cor{
    The category of antisymmetric relations $ARel$ is complete, i.e. has all limits. The limit of reflexive antisymmetric relations is itself
    reflexive and antisymmetric, hence $ARRel$ is a complete subcategory of $ARel$. Finally posets are respected by limits too, since transitivity
    is respected on all components, so $Poset=ARTRel$ is a complete subcategory of $ARRel$.
}
\rem{
    Dually to limits there is a bit more to do, for this consider the best approximation "from the right" of an relation to
    be an antisymmetric relation by compressing the underlying set.
}
\defn{
    For any relation $R\subset S\times S$ introduce its maximal antisymmetric quotient $R^<$ as follows, the underlying set $S^<$ is
    the underlying set $S$ modulo the equivalence relation generated by pairs violating asymmetry:
    \[ S^< = S/(x\sim y \Leftrightarrow xRy \wedge yRx). \]
    Induce a relation on the quotient as before on representatives. The maximal asymmetric quotient of $R$ is reflexive, if $R$ is,
    it trivially satisfies asymmetry. For a transitive relation however that quotient is not necessarily transitive, so potentially
    one application of transitive closure is necessary in the relevant categories, where that is part of the colimit, hence quotient construction.
}
The following insight provides the cornerstone of why colimits of posets look the way they do.
\lem{
    The maximal antisymmetric quotient is left-adjoint to the forgetful functor from asymmetric reflexive relations to reflexive relations:
    \[ (\bullet)^<\colon RRel \Leftrightarrow ARRel \colon U. \]
    \begin{proof}
    The claim is $ARRel(R^<,A)=RRel(R,UA)$. I.e. any morphism from an arbitrary relation uniquely factors through the maximal
    antisymmetric quotient of $R$. This is easily verified. Specifically for the set of all morphisms from any relation to an
    antisymmetric relation to universally satisfy that equation find: A morphism that needs the compatibility condition $xAy$ in
    an antisymmetric relation $A$ coming from an arbitrary relation $R$, can be modified into a morphism that needs $yAx$ satisfied.
    So the quotient introduces all the relations necessary to make $R^<$ a universal source object associated to any reflexive
    relation.
    \end{proof}
}
This lemma gives the first fundamental result of this paper.
\thm{
    The category of antisymmetric reflexive relations $ARRel$ and antisymmetric reflexive and transitive relations, i.e. posets, $ARTRel=Poset$
    are both cocomplete.
    \begin{proof}
    Consider an arbitrary $I$-diagram in $ARRel$, and the colimit in reflexive relations. Then the resulting relation is not necessarily
    antisymmetric, but reflexive automatically. Make it reflexive by passing to the maximal antisymmetric quotient, in the category
    with targets only antisymmetric relations, find that this satisfies the universal property of the colimit in $ARRel$.

    For an arbitrary $I$-diagram in $ARTRel$ take the aforementioned colimit in $ARRel$, by the previous observations that
    colimit is not necessarily transitive on its own, so in addition take the transitive closure, and find that in the
    category $ARTRel=Poset$ this satisfies the universal property of the colimit.
    \end{proof}
}

\rem{
    Reemphasise that fact:
    A limit of posets is just taken componentwise.

    A colimit of posets is given as the componentwise setwise colimit of relations and underlying
    sets, then forcing asymmetry by an equivalence relation, and then extending with transitive closure.
    In particular, all small limits and colimits in $ARRel$ and $ARTRel$ exist, but colimits can potentially collapse
    away a lot of the underlying set.
}

\section{Total orders}
The applications in this paper do not need the full force of well ordering, total orders do just fine, hence totality is the
final property of relations to investigate. To make the constructions natural it is necessary to restrict to specific models
for sets and relations though.
\defn{
    Call a relation $R\subset S\times S$ with underlying set $S$ total, if for each pair $s,t\in S$ at least one of $sRt$ or $tRs$ is satisfied.
    Call the category of total relations with all relation morphisms between them $ToRel$. Analogously define antisymmetric and transitive
    variants $AToRel$, $ToTRel$, and for both $AToTRel$. Also call $AToTRel$ by their more common name: totally ordered sets and in analogy
    with $ARTRel = Poset$ write $AToTRel = Toset$.
}

\rem{
    Note that totality defined in this way trivially implies reflexivity, i.e. $ToRRel = ToRel$.
}
\lem{
    Limits of total relations considered as reflexive relations are total and preserve antisymmetry and transitivity.\\[5pt]
}
\rem{
    Notice that the categories of total relations are far smaller than their non-total counterparts. Among other things total
    relations are "connected", in the sense that for each pair there is a sequence of vertices leading from one element to the
    other through related pairs. With totality that sequence can in fact be chosen to be exactly one relation for each pair.
    In particular, discrete relations cannot be total unless empty or the full relation on a one-point set. Neither can sums of posets
    with two non-empty summands, even if both summands are total posets, i.e. totally ordered sets.
}
\rem{
    This is a triviality by the way the models for categories of relations are set up so far, but with different realisations
    this should be part of an adjunction. Probably $\bar \Delta$ is the forgetful part of that.
}
\prop{
    The category of totally ordered sets contains the prototypical objects $\bar \Delta^S$ with $(S,\leq)$ a total order and
    on the underlying set $S$ for $\bar \Delta^S$ the relation defined exactly by $\leq$.
}
\rem{
    Note the similarity to $\Delta^S$, in particular the fact that without the total order on $S$ there is no way to
    naturally antisymmetrise the construction of the collapsible object without resorting to the maximal antisymmetric quotient.
    Even for $S$ a poset that is not a toset, there is no way to make $\bar \Delta^(\bullet)$ into a functor with respect
    to poset maps, because swapping to maxima always breaks the embedding into a total set.
}
\rem{
    There is a more interesting insight to be had in the same vein though.
}
\prop{
    A totally ordered set is exactly a maximal reflexive, antisymmetric, transitive relation given a fixed underlying set.
    \begin{proof}
    Let $R$ be a total order on the underlying set $S$, assume it were not maximal among the reflexive, antisymmetric
    and transitive relation on $S$ with respect to subset inclusion of relations and identity on the underlying vertex
    set. For $R$ not maximal in $ARTRel$ on $S$ there must exist a pair $(x,y)\in S\times S$ that is not part of $R$,
    but still allows antisymmetry of $R\cup\{(x,y)\}$. However, $R$ is total, so, if not $xRy$, then $yRx$, so follows
    $y=x$ from antisymmetry, so $R$ was already maximal.

    If $R$ were maximal among reflexive, antisymmetric, transitive relations but not total, there would be
    a pair of unequal elements $(x,y)\in S\times S$ which is not $R$-related in either direction. So adding
    the edge $R\cup \{(x,y)\}$ does not violate antisymmetry, since $yRx$ is not satisfied, reflexivity is
    not broken, and using the transitive closure if necessary, it is also transitive. Hence $R\cup \{(x,y)\}$
    generates a strictly greater poset relation on the underlying set, hence $R$ was not maximal.
    \end{proof}
}

\chapter{Pairs of Relations}
\section{Revisiting $ARRel$}
Consider the category of antisymmetric reflexive relations, i.e. relations that apart from a transitive
closure are almost posets. Consider the category of pairs, i.e. inclusions of relations $R\rightarrow U$,
this is the core machine codeable playground section of this paper.

\defn{
    The category $ARRel^2$ of pairs of antisymmetric reflexive relations is defined as follows:
    Objects are antisymmetric reflexive relations $A\subset S_A\times S_A, X\subset S_X\times S_X$
    with underlying sets as indicated,
    such that $S_A\subset S_X$ and $A|_{S_A\times S_A} \subset X|_{S_X\times S_X}$, i.e. the relation
    $A$ is at most the restriction of $X$ to $S_A$.

    Morphisms are pairs of morphisms, i.e. respecting the subset relations.
}


















\section{Putting it all together: Simplicial Complexes and Simplicial Sets}
It is not traditional to impose a total order on the vertices of a simplicial complex. That however yields the
set up for a natural functor to simplicial sets.
\defn{
    Consider $X$ a pair consisting of an underlying totally ordered set, called the vertex set $(VX,\leq_VX)$, and
    a fixed subset of all the finite subsets of $VX$: $\sigma X\colon \mathcal{P}_{fin}(VX)$, called the set of simplices.

    If the sets satisfy $VX = \bigcup \sigma X$, i.e. each vertex is contained in some of the chosen
    subsets, and $\forall t\in \mathcal{P}_{fin}(VX)\colon t\subset s\wedge s\in \sigma X \rightarrow t\in \sigma X$, i.e.
    $\sigma X$ is closed under taking subsets, then call $X = (VX, \leq_VX,\sigma X)$ a simplicial complex, or short a complex.

    A morphism of simplicial complexes is a map of the underlying vertex set, such that for each source simplex the pushed-forward
    subset is a simplex of the target.
}







\chapter{2024-03-18 -- Directed Graphs, Simplicial Complexes and Simplicial Sets}
\section{DAGs}
As a descriptive tool, graphs are invaluable in this paper, hence fix that definition first.
\defn{
    A directed acyclic graph $G = (V,E)$ is a totally ordered set of vertices $V$ together with an edgelist $E$, i.e.
    a chosen subset of non-decreasing pairs of $V$, which includes all edge-touched vertices $v$ as duplicate tuples $(v,v)$ too:
    \[EG\subset \{ ~(v,w)\in V^{\times 2}~|~v\leq_V w~ \wedge (v,v)\in EG \wedge (w,w) \in EG \}.\]

    A morphism of directed acyclic graphs $\varphi\colon G=(VG,EG)\rightarrow(VH,EH)=H$ is a map on vertices $\varphi\colon VG\rightarrow VH$,
    that can be restricted to a map on edges. Clearly $\varphi\colon VG \rightarrow VH$ induces a natural map $\varphi\times \varphi\colon VG\times VG \rightarrow VH\times VH$,
    which can be restricted to $EG$, if it is satisfied that $(\varphi\times\varphi)_*(EG)\subset EH$, call $\varphi$ a morphism of directed acyclic graphs.
}
\rem{
    Since a finite directed graph is acyclic if and only if it admits a topological ordering, this recovers
    the classical notion on finite graphs. Taking colimits over finite subgraphs recovers the total notion.
}
\rem{
    Note however that this definition of morphisms is not the same definition as a lot of algebraic graph theory
    and other references that use homotopies on graphs have. Explicitly, the above definition of morphism allows
    to collapse edges onto vertices, and the graphs have no vertex loops. It is a minor detail, but it changes the
    terminal object, the product structure, etc.
}
\rem{
    The notion "directed acyclic graph" is a classic well-established wording, however, the "acyclic" clashes quite unfortunately with
    the topological applications of this paper, where acyclic has a lot stronger connotations.

    As a consequence, in what follows they are only called {\it dag}s, even just graphs, when it is unambiguous. Undirected graphs,
    graphs with loops, multigraphs, all such variations are not subject of this paper anywhere, hence there should be no confusion.
}

\section{Simplicial Complexes}

Start by fixing the most bird's eye view of how a simplicial complex can be defined.

Note that choosing a total order on the vertices is usually not part of the definition. It is however part of every
example the reader will ever see. It is of essential use in the identification of dags, complexes and simplicial sets,
but plays a silent role in the definition.

\defn{
    A simplicial complex $X=(VX,\leq_{VX},\sigma X)$ is a triple consisting of any set of vertices $VX\in Set$,
    together with a chosen total order $\leq_VX$ on the vertices, and $\sigma X$ a fixed subset of finite subsets
    of $VX$, which is closed under taking subsets. I.e. any subset of the vertices $VX$ that is a subset
    of an element of $\sigma X$ is itself an element of $\sigma X$:
    \[\forall S\subset VX \colon (\exists T \in \sigma X\colon S\subset T) \Rightarrow S \in \sigma X.\]
    Call an $(n+1)$-element subset (canonically ordered by restricting $\leq_{VX}$)
    $(v_0<_{VX}\ldots<_{VX}v_n)=t\in\sigma X$ an $n$-simplex of $X$, and hence call $\sigma X$ the set of $X$'s simplices.
    In addition assume for each $v\in VX$ that its associated $0$-simplex is actually a simples $\{v\}\in\sigma X$.

    A morphism of simplicial complexes $\varphi\colon X = (VX,\leq_{VX},\sigma X)\rightarrow Y = (VY,\leq_{VY},\sigma Y)$, is
    a set map $\varphi\colon VX \rightarrow VY$, which respects the chosen simplex sets:
    \[ \forall s\in \sigma X\colon s = (s_0<\ldots<s_n) \Rightarrow \varphi_*(s)=\{ \varphi(s_i) \}_{0\leq i\leq n} \in \sigma Y, \]
    where $\varphi_*(s)=\{ \varphi(s_i) \}_{0\leq i\leq n}$ is a subset of the $VY$ vertices, which is potentially strictly
    smaller than $s$, if a vertex is repeated.
}
\rem{
    Note that assuming vertices are $0$-simplices is not a loss of generality. Because for a pair $X=(VX,\sigma X)$,
    satisfying the other properties, the set $\bigcup \sigma X$ that is the union over all the finite subsets appearing in $\sigma X$
    is a reduced vertex set satisfying that assumption.

    In particular a morphism of simplicial complexes is uniquely determined by its restriction to either vertices or simplices.
}
\rem{
    Given a set map of vertices $\varphi_V\colon VX\rightarrow VY$, it induces a map of finite subsets by applying $\varphi$ elementwise,
    hence covariantly: $\varphi_*\colon \mathcal{P}(VX) \rightarrow \mathcal{P}(VY)$.

    For $\sigma X \subset \mathcal{P}(VX)$ and $\sigma Y \subset\mathcal{P}(VY)$, $\varphi$ is a simplicial map if and only if, restricting
    to $\sigma X$ has its image in $\sigma Y$, i.e. $\varphi_*$ induces a well-defined map of the simplices by restriction $\varphi_*\colon \sigma X \rightarrow \sigma Y$.
    Restricting to $\sigma X$ is not a condition, being able to restrict to $\sigma Y$ is the salient point.
}
\rem{
    Since the content of this paper is about dags, simplicial complexes and simplicial sets, occasionally emphasise them apart as: dags, complexes, spaces.
}

\section{Complexes and DAGs}
\thm{
    Restricting a simplicial complex $X$ to its vertices and $1$-simplices defines a dag $sk_1X=(VX,\{~ e \in \sigma X~|~ |e|\leq 2 ~\}$.
    Any directed acyclic graph $G=(VG,EG)$ in turn defines a simplicial complex, its flag complex $Flag(G)=(VG,\sigma(G))$ with $\sigma G$ the finite subsets
    of $VG$ such that for each pair in that subset the edge is part of $EG$:
    \[ Flag(G):=\{~\{(v_0\leq_{VG} v_1 \leq_{VG} \ldots \leq_{VG} v_n)\} \in VG^* ~|~ \forall i\leq j\colon (v_i,v_j)\in EG ~\}. \]
    Each of these are functors, i.e. maps of simplicial complexes induce dag-maps on the $1$-skeleton and maps of dags induce simplicial complex maps on the
    flag complexes, respecting composition and identities covariantly.
    \begin{proof}
    Given a simplicial complex $(VX,\leq_{VX},\sigma X)$ define the vertex set as just the same vertex set $VX$ and the edgelist as the simplices
    of dimension less than or equal to $1$. The total order on $VX$ makes these simplices into non-decreasing tuples unambiguously.
    Clearly a simplicial map in particular induces a map of the $1$-skeletons, since it at most reduces degrees, but never increases a simplex degree.
    Hence follows also the composition and identity property.

    Given a dag $(VG,\leq_VG,EG)$ define the vertex set as just the same vertex set $VG$,
    and for each sequence of edges that are transitively closed in $EG$ add a simplex of appropriate size in $\sigma Flag(G)$:
    \[\sigma Flag(G) := \{ ~\{v_0\leq_VG v_1 \leq_VG \ldots \leq_VG v_n \} ~|~ \forall i\leq j\colon (v_i,v_j)\in EG ~\},\]
    where $\{ v_0,v_1,\ldots,v_n\}$ is the set underlying the tuple $(v_0,\ldots,v_n)$, which could potentially have fewer elements than
    the tuple because our edgelist includes the vertices too $(v,v)$. In particular it is trivially satisfied that $\sigma Flag G$ is closed
    with respect to subsets. The tuple $Flag(G)=(VG,\leq_VG,\sigma Flag G)$ hence defines simplicial complex.

    A map of vertices that respects edges clearly induces a map of flag complexes, the identity induces the identity, composition is respected covariantly.
    \end{proof}
}
\thm{
    Taking the $1$-skeleton of a $Flag$ complex of a dag naturally is exactly the same dag set wise. I.e. \[ sk_1\circ Flag = id_{DAG}. \]
    \begin{proof}
    Consider a dag $(VG,EG)$, then its flag complex consists of simplices the vertices $VG$ and flags defined by $EG$. The $1$ skeleton of the
    flag complex consists of the vertices, which are $VG$, and the $1$-flags, i.e. pairs of vertices $(v_0,v_1)$ such that $(v_0,v_1)\in EG$, so that
    is exactly $EG$.
    \end{proof}
}
\thm{
    The $Flag$ complex of the $1$-skeleton of a simplicial complex has a natural inclusion from the original complex. I.e.
    \[ \exists \eta\colon Id_{sCx}\Rightarrow Flag\circ sk_1. \]
    \begin{proof}
    A map of simplicial complexes is uniquely determined by what it does on vertices. Do note that $Flag(sk_1(X))$ and $X$ have
    the same vertices by definition, hence it remains to check, that the identity induces a simplicial map, i.e. respects simplices.

    Given any simplex $s=(s_0<\ldots<s_n)\in \sigma X$, note that all the subsets $(s_i,s_j)$ are part of the $1$-skeleton, hence part
    of the edges of the $1$-skeleton. So by constructing the flag complex $s$ is in particular an element of $\sigma Flag(sk_1(X))$.
    \end{proof}
}
\ex{
    Be very aware that this is usually a proper inclusion. Consider the simplicial complex $(\{0,1,2\},\{0,1,2,01,02,12\})$. Reducing to
    the $1$-skeleton is basically the same pair, but extending via flag complex induces the simplex $\{012\}$ that was not there before.

    In particular from a topological point of view this is the inclusion of a circle into a disc, so very much not an equivalence in any
    desirable sense.
}

There is however a big class of simplicial complexes for which the reverse direction is the identity too, for which the following definition
is needed.

\defn{
    Given a simplicial complex $(VX,\leq_VX,\sigma X)$, define its subdivision $sd(X)$ as follows, the vertices are $V_{sd(X)}=\sigma X$, with total
    order induced by ordering the (finite!) tuples first by length, then order ordered tuples lexicographically according to $\leq_VX$,
    inducing a total order on $V_{sd(X)}$. The simplices are all chains of
    vertices $(v_0,v_1,\ldots,v_n)\in (\sigma X)^{n+1}$ such that $v_i \subset v_j \forall i\leq j$.
}

Subdivided simplicial complexes actually have a defining dag.

\prop{
    The subdivision of a simplicial complex is in fact the flag complex of a defining dag on vertices $\sigma X$, totally ordered as above, and
    edges included according to subset relation, identity in particular.
    \begin{proof}
    Note the ordering of vertices of $sd(X)$ implies that each edge of $sd(X)_1$ is in fact
    non-decreasing with respect to that length and lexicographic order. The vertices of that graph are the totally ordered set $\sigma X$. It is
    easy to see that the flag complex of this graph is exactly the definition above.
    \end{proof}
}

Hence because one side of the composition strictly equals the identity get that:


\prop{
    On the category of simplicial complexes which are at least once subdivided the flag complex of the $1$-skeleton yiels the exact same complex
    $Flag\circ sk_1 \circ sd = id_{sCx}\circ sd$. I.e. the whole simplicial complex can be exactly setwise recovered from its vertices and edges.
}


In fact the category of simplicial complexes which are equal to their flag complex is far bigger, one has to exclude only the above mentioned example
in a universal manner.

\prop{
    Let $(VX,\leq_VX,\sigma X)$ be a simplicial complex. Assume additionally it satisfies on $1$-simplices
    \[\forall u\leq_{VX} v\leq_{VX} w \in VX\colon (u,v),(v,w),(u,w) \in \sigma X \Rightarrow (u,v,w) \in \sigma X.\]
    Call the simplex $(u,v,w)$ a witness for the transitivity of $u,v,w$.

    Then $X$ is the flag complex of its $1$-skeleton.
    \begin{proof}
    By induction the aforementioned condition in fact implies that every such flag of vertices with pairwise edges between
    them gives rise to a simplex in $\sigma X$ already. In particular, every simplex of $\sigma X$ is of that form. Hence $X$ is the flag complex of its $1$-skeleton, its
    edgelist.
    \end{proof}
}
\cor{
    The subdivision satisfies the above condition, hence the identity $Flag(sk_1(sd(\bullet))) = sd(\bullet)$ follows as a corollary to the above proposition, too.
}
\ex{
    Note that there are basically two ways in which a simplicial complex can satisfy the condition. It is about either thin paths or transitive closures. Consider one extreme:
    \[ (0123,\{0,1,2,3,01,12,23,03\}), \]
    this is a path from $0$ to $3$ together with a direct shortcut edge $03$. It is the flag complex of its $1$-skeleton, since for no two pairs of composable edges $(u,v),(v,w)$
    there is a third pair $(u,w)$ which would necessitate a $2$-simplex witnessing the transitivity $(u,v,w)$.

    On the other extreme consider a full simplex, i.e. \[ (012\ldots n,\mathcal{P}(\{0,1,\ldots,n\})). \]
    Then obviously for each triple of edges that should be transitive, there is a $2$-simplex witnessing that transitivity, so the flag complex introduces no new simplices.
}
\section{Spaces - Simplicial Sets}
Introduce simplicial sets in a manner that is compatible with how dags and complexes are fixed.
\defn{
    Fix a pair of a sequence of totally ordered sets indexed over finite totally ordered sets $X=((X_S)_{S\in Fin_\leq},\{\phi^*\})$ together with for each map of finite totally ordered
    sets $\phi\colon (S,\leq_S) \rightarrow (T,\leq_T)$ a chosen "induced map" in the opposite direction $\phi^*\colon X_T \rightarrow X_S$.

    If the chosen $\phi^*$ satisfy the contravariant functoriality condition \[\phi^*\psi^* = (\psi\circ\phi)^*~~\wedge~~ id_S^* = id_{X_S},\] call the pair of sets and induced maps
    $X=((X_S)_{S\in Fin_\leq},\{\phi^*\})$ a simplicial set or a space, and the elements $x\in X_S$ the $S$-simplices of $X$.

    A morphism of simplicial sets $f\colon X\rightarrow Y$ is a degreewise set-map $f_S\colon X_S \rightarrow Y_S$, i.e. a map of $S$-simplices for each finite totally ordered set $S$,
    which is compatible with the induced maps $\phi^*$ on $X$ and $Y$, i.e.
    $\forall S,T \forall \phi_\leq\colon S\rightarrow T, f_S\circ\phi^*=\phi^*\circ f_T.$
}
\rem{
    Note that the choice of morphisms legitimises writing $\phi^*$ instead of the more explicitly descriptive $\phi^*_X$ and $\phi^*_Y$.

    Note also again, just like for dags and complexes, each set of $S$-simplices is chosen to be totally ordered, but it did not enter the definition in any other way.
    It serves to set up the correspondence between simplicial sets and simplicial complexes.
}
\defn{
    For a simplicial set $X = ((X_S)_S, \{~\phi^*\colon X_T\rightarrow X_S ~|~\phi \in Fin_\leq(S,T)~\}_{S,T}$ and $x\in X_S$ any $S$-simplex and $i\colon \bar S \rightarrow S$ a proper injection
    of finite totally ordered sets, call such a $i^*x$ a face of $x$. For $x\in X_S$ any $S$-simplex and $p\colon \bar S \rightarrow S$ a proper surjection call $p^*x\in X_{\bar S}$ a degeneracy
    of $x$.

    For the specific finite totally ordered sets $\lbrack n \rbrack := \{0,\ldots,n\}\subset (\N,\leq_\N)$ ordered as subsets of the natural numbers, simply call the $\{0,\ldots,n\}$-simplices
    $n$-simplices, and write: $X_n := X_{\{0,\ldots,n\}} = X_{\lbrack n \rbrack}$.
}
\rem{
    Note that a simplicial set is uniquely determined by its $n$-simplices given for each $n\in \N$, since each finite totally ordered set has a unique isomorphism to a set of the form $\lbrack n \rbrack$.
    Hence the induced maps of the simplicial provide the rest of the structure on arbitrary finite totally ordered sets. In particular, it is natural to assume the total order on $n$-simplices is the
    same as on $S$-simplices for all other $n$-element sets $S$ identified along the canonical isomorphism $(S,\leq_S)\rightarrow (\lbrack n\rbrack,\leq_\N)$.
    Hence a simplicial set can be defined only by defining its $n$-simplices.
}
\section{Complexes are Spaces - Simplicial Complexes are special Simplicial Sets}
\subsection{Simplicial Complexes are Simplicial Sets}
This is how the total orders enter, with these definitions a simplicial complex is just a special type of simplicial set.
\lem{
    Each simplicial complex $X$ has a unique naturally associated simplicial set $sX$, in particular, morphisms of simplicial complexes
    naturally induce morphisms of their associated simplicial sets.
    \begin{proof}
    Let $X=(VX,\leq_VX,\sigma X\subset \mathcal{P}(VX))$ be a simplicial complex, i.e. $VX$ the set of vertices,
    totally ordered by $\leq_VX$ and $\sigma_X$ a subset-closed subset of the finite subsets of $X$-vertices $VX$.

    Define a simplicial set $sX$ associated to $X$ as follows on $n$-simplices only: \[ sX_n:= \{~s\in VX^{\times (n+1)}~|~ \forall i\leq j\colon s_i\leq s_j \wedge \{s_i\}_{0\leq i\leq n} \in \sigma X ~\}. \]
    I.e. $n$-simplices are $VX$-ordered $n+1$-tuples $s$ of $VX$, such that the set with duplicates removed $\{s_i\}_{0\leq i\leq n}$ is part of the original $\sigma X$ simplices of $X$.

    To define the induced maps for $X$, given a map of finite totally ordered sets $\phi\colon \lbrack m \rbrack \rightarrow \lbrack n\rbrack$,
    and an element $s\in sX_n = \{~s\in X^{\times (n+1)}~|~\forall i\leq j\colon s_i\leq s_j \wedge  \{s_i\}_{0\leq i\leq n} \in \sigma X ~\}$.
    By applying the set-map $\phi$ on indices $s = (s_0<\ldots<s_n)$ define $\phi^*(s) = \{s_{\phi(i)}\}$, i.e. the subset of $\sigma X$
    defined by taking the $\phi$ image of $s$ according to its tuple/$\leq_VX$-indices.

    It easily follows that $id^* = id$ and $(\psi\varphi)^*=\varphi^*\psi^*$, hence $sX$ is a simplicial set. Furthermore a morphism of simplicial complexes
    evidently induces a compatible morphism of these simplicial sets. The identity if started with the identity, and respecting compositions covariantly.
    \end{proof}
}
\subsection{Simplicial Complexes are a specific subcategory of Simplicial Sets}
The fun part is to go back! For that as a first step identify which simplicial sets come from simplicial complexes.

Note that the condition on simplicial sets basically recalls the well-known notion that a morphism of simplicial complexes is uniquely determined
by what it does on vertices, which is part of our definition too.
\thm{
    Let $X = ((X_S)_S, \{~\phi^*\colon X_T\rightarrow X_S ~|~\phi \in Fin_\leq(S,T)~\}_{S,T}$ be a simplicial set.

    If the simplices of $X$ satisfy that having all faces equal already implies equality for each pair of simplices, then $X$ is a simplicial set constructed like above.
    I.e. if for all $x,y\in X_S$ and for all proper injections $i\colon T\rightarrow S$ the simplices $x$ and $y$ satisfy $i^*x=i^*y$, then $X$ is naturally isomorphic
    to a naturally associated simplicial complex $\bar X$ with associated simplicial set $s\bar X\cong X$.
    \begin{proof}
    For $X$ a simplicial set define the vertex set $VX = X_0 = X_{\{0\}}$ the $0$-simplices of $X$. Induce the total order from the total order on $0$ simplices.

    For $j_i\colon 0 \rightarrow n$ the (trivially order preserving) map that sends $0$ to $i$, defined for each $0\leq i \leq n$, note
    that the uniqueness condition means, one can uniquely describe an $n$-simplex $x\in X_n$ with the set of its vertices $(j_0^*x,j_1^*x,\ldots,j_n^*x)$.
    Note however that not every such tuple of vertices is a simplex in $X$ obviously.

    So define $\sigma \bar X := \{~ \{ (j_0^*x,j_1^*x,\ldots,j_n^*x) \} \in VX^{\times n} ~|~\forall n\in\N,  x\in X_n. ~\}$.
    I.e. define the simplicial complex $\bar X$ to be given by the $0$-simplices of $X$ as vertices, totally ordered as in $X$, and higher simplices are exactly the
    subsets of vertices that can be written as induced by $j_i$ tuples like above. Since the set $\{j_i\}$ is a generator in the category of finite totally ordered
    sets, it actually implies $i^*x = i^*y$ if $(j_0^*x,j_1^*x,\ldots,j_n^*x)=(j_0^*y,j_1^*y,\ldots,j_n^*y)$, because each injection can be reconstructed from including
    points one by one inductively.

    Clearly the subset condition on $\sigma \bar X$ is satisfied, by considering restriction of a simplex along inclusions $\lbrack n-1 \rbrack \rightarrow \lbrack n \rbrack$, i.e. faces.

    Consider the simplicial set that $\sigma \bar X$ generates according to the lemma immediately above. Clearly on vertices the identity is a natural map on level $0$, and
    note that the condition implies exactly that one can uniquely reconstruct any $n$-simplex from its vertex-tuple like above, so the map of simplicial sets
    \[ j_* \colon X \rightarrow s\bar X \]
    defined by sending an $n$-simplex $x$ to $(j_0^*x,j_1^*x,\ldots,j_n^*x)$, is evidently compatible with the respective induced maps from finite
    ordered sets, because set maps are defined pointwise. It is injective in each degree by assumption and trivially surjective by construction
    of $\sigma \bar X$. Hence $X$ is naturally isomorphic to the simplicial set of the simplicial complex $\bar X$, thus follows the claim.
    \end{proof}
}
\cor{
    This implies in particular that for a simplicial set, which is a simplicial complex in the sense of the above theorem, all its simplicial subobjects are
    automatically simplicial complexes too, because the condition gets easier to satisfy on subsets.
}
\rem{
    Note that the correspondence over the maps $j_i$ allows to assume without loss of generality that simplices of a simplicial complex are given as such
    tuples of vertices. For general simplicial sets a priori this is not a requirement and also not helpful, because simplicial sets are precisely about allowing
    multiple simplices with the same faces, hence they can usually not be represented like this.
}
In particular it follows in the category of simplicial sets too:
\thm{
    Any morphism from the simplicial set associated to a simplicial complex to any other simplicial set is uniquely determined
    by what it does on vertices / $0$-simplices. % strong indication for X -> VX left adjoint to something
}
\section{Simplicial Sets can be resolved by a pair of Simplicial Complexes}
\thm{
    Let $X = ((X_S)_S, \{~\phi^*\colon X_T\rightarrow X_S ~|~\phi \in Fin_\leq(S,T)~\}_{S,T}$ be a simplicial set. Then
    there is an inclusion of simplicial complexes $A\rightarrow Z$, which as a quotient in simplicial sets exactly yields $X$.
    \begin{proof}
    By (if necessary transfinite) induction, assume finitely many non-degenerate simplices in $X$. I.e. consider each simplicial set
    as the colimit over its subsets with finitely many non-degenerate simplices, the finite steps that follow still apply consistently.


    \end{proof}
}
\section{}
\thm{
    \[ sFlag(sk_1(sd^2(X))) = sd^2X \]
    for all simplicial SETS $X$.
}

\end{document}