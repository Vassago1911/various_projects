% This file is part of beamerthemepureminimalistic.

% If problems/bugs are found or enhancements are desired, please contact
% me over: https://github.com/kai-tub/latex-beamer-pure-minimalistic

\documentclass[a4paper]{book}
\include{version}
\usepackage{amsmath,amssymb,amsthm}
\newtheorem{thm}{Theorem}[section]
\newtheorem{theorem}{Theorem}[section]
\newtheorem{lem}[thm]{Lemma}
\newtheorem{lemma}[thm]{Lemma}
\newtheorem{prop}[thm]{Proposition}
\newtheorem{proposition}[thm]{Proposition}
\newtheorem{cor}[thm]{Corollary}
\newtheorem{corollary}[thm]{Corollary}
\theoremstyle{definition}
\newtheorem{Remi}[thm]{Reminder}
\newtheorem{rem}[thm]{Remark}
\newtheorem{defn}[thm]{Definition}
\newtheorem{ex}[thm]{Example}
\newtheorem{nonex}[thm]{{\bf NON}-Example}
\newtheorem{conj}[thm]{Conjecture}
\newtheorem{Exercise}[thm]{Exercise}
\newtheorem{Code}[thm]{Pseudocode}
\usepackage{datetime2}
\usepackage{graphicx}
\usepackage{svg}
% \svgpath{{/home/vassago/LATEX/}{/home/vassago/LATEX/svgs/}}
\usepackage{svg-extract}
\usepackage{hyperref}
\usepackage{inconsolata}

\usepackage{tcolorbox}

\usepackage[all]{xy}
\usepackage{xypic}
\usepackage{xcolor}

\definecolor{light}{RGB}{0,100,43}
\definecolor{dark}{RGB}{130,0,100}
\definecolor{exercise}{RGB}{220,200,255}
\definecolor{textcolor}{RGB}{0, 150, 128}
\definecolor{title}{RGB}{150, 0, 128}
\definecolor{footercolor}{RGB}{0, 128, 64}
\definecolor{code}{RGB}{240,220,160}
\definecolor{codebg}{RGB}{50,50,50}
\definecolor{mathbg}{RGB}{0,10,30}
\definecolor{mathtext}{RGB}{220,220,250}
\definecolor{bg}{RGB}{32, 0, 16}

\def\exercise#1{\color{light}{\h\Exercise{\color{exercise}\scshape #1}}\color{textcolor}}
\def\code#1#2{\color{dark}{\h\Code{\h}\\[.5em]%
\color{mathtext}#1
\begin{tcolorbox}[colback=codebg]%
\color{code}%
\texttt{#2}%
\color{textcolor}%
\end{tcolorbox}\color{textcolor}}}

\def\backtick{`}
\def\curtime{\DTMdate{2023-09-21} ~\DTMtime{01:42:17}\DTMdisplayzone{-5}{00}}
\def\light#1{{\color{light}#1}}
\def\dark#1{{\color{dark}#1}}
\def\h{\hspace{1em}}

\def\signature{\href{mailto:marc@lange-data.org}{Dr. Marc Lange, marc@lange-data.org}}
\def\mailsignature{\href{mailto:marc@lange-data.org}{marc@lange-data.org}}

\def\abs#1{\left| #1 \right| }
\def\floor#1{\lfloor #1 \rfloor }

\newcommand{\lthree}{<\hspace{-5.2pt}3}
\newcommand{\E}{\mathcal{E}}
\newcommand{\DAG}{\mathcal{DAG}}
\newcommand{\Z}{\mathbb{Z}}
\newcommand{\R}{\mathbb{R}}
\renewcommand{\P}{\mathfrak{P}}
\newcommand{\Q}{\mathbb{Q}}
\newcommand{\N}{\mathbb{N}}
\renewcommand{\S}{\mathbb{S}}
\newcommand{\C}{\mathcal{C} }
\newcommand{\D}{\mathbb{D}}
\newcommand{\LL}{\mathcal{L}^1(\Z,\Z)}
\newcommand{\Set}{\mathrm{Set}}

\renewcommand{\dag}{\mathrm{emb}\mathcal{DAG}}


\newcommand{\s}{Set^{\phantom{\leq}}_{\phantom{+}}}
\newcommand{\spp}{Set=fSet}
\newcommand{\smp}{Set^{\leq}_{\phantom{+}}}
\newcommand{\spm}{Set^{\phantom{\leq}}_{+}}
\newcommand{\smm}{Set^{\leq}_{+}}
\newcommand{\spmpm}{Set^{\pm}_{\pm}}

\newcommand{\App}{A}
\newcommand{\Amp}{A^{\leq}_{\phantom{+}}}
\newcommand{\Apm}{A^{\phantom{\leq}}_{+}}
\newcommand{\Amm}{A^{\leq}_{+}}

\begin{document}
\chapter{VR complexes, Flags and DAGs}
\section{basics}
\defn{DAG - vertices $=\lbrack n\rbrack$}
\defn{VR - $VR_\varepsilon(X)$ wrt max norm}
\prop{$VR = Flag(sk_1(VR(X)))$}
\section{Each dag is a VR complex}
\defn{
    A VR embedding of a directed acyclic graph $G=(VG,EG)$ is a
    collection of points $X=\{x_1,\ldots,x_{|G|}\}$ and a fixed
    $\varepsilon > 0$, with a map $\iota_G\colon VG \rightarrow \R^2$ with image $X$,
    such that the Vietoris-Rips-Complex recovers the graph:
    \[ VR_\varepsilon(X)=\{~ (x_i,x_j) \in X^2 ~|~i\leq j, d(x_i,x_j)<\varepsilon ~\}  \]
    \[ \cong (VG,EG). \]
}

\lem{
    For a disconnected graph $G=H_1\sqcup H_2$ which has a VR embedding
    for each summand $H_i$, the sum $G$ has a VR embedding as well.
    \begin{proof}
    Given two such embeddings by the assumptions one only has to make sure
    no edges are introduced between $H_1$ and $H_2$. To do so consider their
    isomorphisms $VR_{\varepsilon_1}(i_1H_1)\cong H_1$ and $VR_{\varepsilon_2}(i_2H_2)\cong H_2$.
    One convinces oneself easily that by scaling just one of the embeddings, it
    can be arranged that $\varepsilon := \varepsilon_1 = \varepsilon_2$ without
    loss of generality. By embedding both (compact, hence bounded) images in $\R^2$
    at more distance than $\varepsilon$, a VR complex with the same or smaller $\bar\varepsilon$
    recovers the disjoint sum $G = VR_\varepsilon(i_1H_1\sqcup i_2H2)$ for $\sqcup$ just
    a suggestive notation for the embedding of the summands at "great distance".
    \end{proof}
}

\lem{
    Every fully connected graph has a VR embedding.
    \begin{proof}
    Count the vertices $G = \{1,\ldots,n\}$ with $n\geq 0$ and embed them as
    $X=\{(i,0)|1\leq i \leq n\}$. Then $VR_{n+\varepsilon}(X) \cong G$ for any
    $\varepsilon >0$.
    \end{proof}
}

\lem{
    Consider a subgraph in a graph $M\subset G$ defined by a restricted vertex set $VM$
    and all induced edges of $G$. Additionally for each partition of
    $VG = V_1 \sqcup VM \sqcup V_2$, consider the induced
    subgraphs $G_1 = G\lbrack V_1 \sqcup VM\rbrack$ and $G_2 = G\lbrack V_2 \sqcup VM\rbrack$.

    If $G_1$ and $G_2$ each already see all of $G$'s edges, i.e. if $EG_1\cup EG_2 = EG$: Then $G$ is the categorical pushout of $G_1$ and $G_2$ over $M$.
    \begin{proof}
    Given a map $G\rightarrow Z$ for any graph $Z$ there clearly are uniquely induced maps
    restricted to $G_1$ and $G_2$, which are compatible over $M$.

    Given a pair of maps $\varphi_i\colon G_i \rightarrow Z$ that agree on $M$, show that there
    is a unique map $\varphi\colon G\rightarrow Z$ inducing the $\varphi_i$.

    By assumption on the vertex sets $\varphi$ is clearly uniquely determined, so one has
    to show that the map induced by the map on vertex sets is in fact a dag map.
    This is true if and only if there are not any edges connecting $V_1$ and $V_2$, otherwise
    one can universally construct an example of a dag map on both subsets, agreeing on the
    intersection, that breaks that edge. By assumption on the edge-sets, this cannot happen,
    any edge in $G$ is either part of $G_1$ or $G_2$ or both, hence of $M$. So all edges
    of $G\rightarrow Z$ in fact are edges in $Z$ as well, hence get the unique induced
    map of the pushout.
    \end{proof}
}

\lem{
    Let $G = VR_1(X)$ for some finite point set $X\subset \R^2$, then $X$ can be perturbed
    in such a way that $EG = \{~(v,w)\in X\times X~|~v=w \vee
    ( \varepsilon_l<d(v,w)<\varepsilon_r \wedge d(v,w) = |v_x-w_x| )~\}$.
    In words: The edge set are the pairs of nodes which are close enough, and all of them
    have their maximal distance to each other in the $x$-component.
    \begin{proof}
    If $G$ is arranged so that all distances are instead realised by $y$-components, just
    transpose $\tau\colon \R^2 \rightarrow \R^2$. The resulting image satisfies the $x$-component
    condition then. If $G$ is arranged already satisfying the conditions, there is nothing
    to prove. Hence consider the set of all vertices such that all their distances to other
    vertices are realised in each case by the $x$-component, the sets of all vertices which
    realise each case by their $y$-component, and the set of vertices which are in neither other
    set, i.e. which have one vertex which realises their distance with $x$ and another with $y$.

    If the $x$-component set is larger than one vertex, .. ?
    If the $y$-component set is larger than one vertex, .. ?

    \end{proof}
}

\cor{ % <=> ein VR complex acc to max norm kann so gebaut sein, dass das max immer von
      %     derselben koordinate realisiert wird
    A VR embedding of a graph can always be modified to have its vertices embedded
    in strictly $y$-coordinate-increasing order.
    \begin{proof}
    This is clear for the empty graph, the one-point graph, any graph on two points.
    So assume the graph has at least $3$ nodes which are already without loss of
    generality enumerated points in $\R^2$: $VG = X = \{~(x_i,y_i)\in \R^2~ |~1\leq i\leq n~\}$
    with edgelist $EG = \{~ (v,w) \in X\times X~|~ d(v,w)<2 ~\}$. I.e. the vertex
    set is a set of points in the 2d plane with an edge between two points if and
    only if their distance is less than 2.

    Choose the distance $d(v,w)=\lVert v - w \rVert_\infty$, which is also no loss
    of generality since $\R^2$ is a finite-dimensional vector space. This results in
    the edge list becoming:
    \[ EG = \{~((v_x,v_y),(w_x,w_y))~|~0\leq |v_x - w_x| < 2 \wedge 0 \leq |v_y-w_y| < 2~\}. \]
    Since the graph is finite there is also a non-zero minimum distance of the nodes
    \[ \forall v,w \in X\colon v\neq w \Rightarrow d(v,w)>\varepsilon. \]
    By rescaling again assume without loss of generality the edge list is:
    \[
    \begin{aligned}
    EG = & \{~((v_x,v_y),(w_x,w_y))~|~\\
    & 0 \leq |v_x - w_x| < K \\
    &\wedge 0 \leq |v_y-w_y| < K \\
    & \wedge 1 < min(|v_x-w_x|,|v_y-w_y|) ~\}\cup \{(v,v)|v\in X\}
    \end{aligned}
    \]
    for some natural number $K\in \N$.
    % TODO: conditions auf  1< bla < 2 homogenisieren, pr_0 und pr_1 injektiv machen

    % ODER: jump right in?
    If the index maximal node in $X$ is already the highest $y$-coordinate in the embedding,
    inductively rearrange the nodes below it, and the claim follows.

    Hence assume the maximal node is not $y$-maximally embedded, and fix that.
    Cover $G$ as $G\setminus y \cup \bar y$ for $\bar y$ the graph on vertices $X$
    with edges
    \[ E \bar y = \{~(a,b)\in X~|~b=y\wedge a\neq y \wedge
    0\leq \lVert a-y\rVert < K \wedge 1 < min(|a_x-y_x|,|a_y-y_y|) < K ~\}. \]

    \end{proof}
}

\cor{
    A VR embedding of a graph can be modified to have the vertices embedded in any
    arbitrary order.
    \begin{proof}
    Relabel the graph according to the vertex permutation and direct it according
    to the new labels. By the result above it can be embedded in increasing order,
    but the permutation is arbitrary, hence follows the result.
    \end{proof}
}

\rem{
    The crucial point to inductively construct VR embeddings for all graphs is
    the following lemma.
}

\lem{
    Given a VR embedding of a graph $G\rightarrow \R^2$, and a binary partition
    of the vertices $VG = V_1 \sqcup V_2$, the embedding can be perturbed so that
    the vertices satisfy:
    \[ \forall v \in V_1 \forall w \in V_2\colon d(v,w)>\varepsilon, \]
    for some $\varepsilon > 0$.
    \begin{proof}
    Assume without loss of generality the graph is a VR complex with radius $1$:
    $VR_1(X)$ with 
    \end{proof}
}

\thm{Each dag $G$ with finitely many vertices is a $sk_1(VR(X))$.
\begin{proof}
I.e. it is to show that for each finite dag $G$ there is a $2d$ point
configuration of finitely many points with $sk_1(VR(X)) = G$. If that
is to be true, assume wlog $VG = X$. By compactness assume wlog
$X = \{~(x_i,y_i)\in \lbrack 0,n\rbrack \times \lbrack 0,1\rbrack~|~1\leq i\leq |X|=n,~\}$
such that $min_i(x_i) = x_0 = 0$ and $max_i(x_i) = x_n = n$.

For $G=(X,EG)$ with $|X|\leq 3$ it is easy to see all dags are VR complexes:
The empty graph comes from the empty set, the one-point graph from any one-point
set, the disconnected graph on two points can be realised with $X$ satisfying the above with $\varepsilon < \frac12$ and the connected graph on two points can be realised with $\varepsilon > \frac12$.
For $|X|=3$ the maximally connected case is a triangle, so arrange $X = \{(0,0),(0.5,0),(1,0)\}$, then
for $\varepsilon > 1$ the complex $VR_\varepsilon(X)$ is fully connected on two vertices. If there is any
edge not connected in the original graph on $3$ vertices, one can see how to push two $\R^2$ points apart
so that they stay connected to a third point, but not to each other. If there is another edge missing,
push the third point closer to one extremal point, so it leaves the range of the other point. If there
are not any edges in the original graph..

% <- grauenhaftes stueck I0, das muss doch erhellender gehen

Continue by induction, consider the $x\in X$ with the maximal $pr_0(x)$-value, call it $x=(x_0,x_1)$.

\end{proof}}

\end{document}