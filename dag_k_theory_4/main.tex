% This file is part of beamerthemepureminimalistic.

% If problems/bugs are found or enhancements are desired, please contact
% me over: https://github.com/kai-tub/latex-beamer-pure-minimalistic

\documentclass[a4paper]{book}
\include{version}
\usepackage{amsmath,amssymb,amsthm}
\newtheorem{thm}{Theorem}[section]
\newtheorem{theorem}{Theorem}[section]
\newtheorem{lem}[thm]{Lemma}
\newtheorem{lemma}[thm]{Lemma}
\newtheorem{prop}[thm]{Proposition}
\newtheorem{proposition}[thm]{Proposition}
\newtheorem{cor}[thm]{Corollary}
\newtheorem{corollary}[thm]{Corollary}
\theoremstyle{definition}
\newtheorem{Remi}[thm]{Reminder}
\newtheorem{rem}[thm]{Remark}
\newtheorem{defn}[thm]{Definition}
\newtheorem{ex}[thm]{Example}
\newtheorem{nonex}[thm]{{\bf NON}-Example}
\newtheorem{conj}[thm]{Conjecture}
\newtheorem{Exercise}[thm]{Exercise}
\newtheorem{Code}[thm]{Pseudocode}
\usepackage{datetime2}
\usepackage{graphicx}
\usepackage{svg}
% \svgpath{{/home/vassago/LATEX/}{/home/vassago/LATEX/svgs/}}
\usepackage{svg-extract}
\usepackage{hyperref}
\usepackage{inconsolata}

\usepackage{tcolorbox}

\usepackage[all]{xy}
\usepackage{xypic}
\usepackage{xcolor}

\definecolor{light}{RGB}{0,100,43}
\definecolor{dark}{RGB}{130,0,100}
\definecolor{exercise}{RGB}{220,200,255}
\definecolor{textcolor}{RGB}{0, 150, 128}
\definecolor{title}{RGB}{150, 0, 128}
\definecolor{footercolor}{RGB}{0, 128, 64}
\definecolor{code}{RGB}{240,220,160}
\definecolor{codebg}{RGB}{50,50,50}
\definecolor{mathbg}{RGB}{0,10,30}
\definecolor{mathtext}{RGB}{220,220,250}
\definecolor{bg}{RGB}{32, 0, 16}

\def\exercise#1{\color{light}{\h\Exercise{\color{exercise}\scshape #1}}\color{textcolor}}
\def\code#1#2{\color{dark}{\h\Code{\h}\\[.5em]%
\color{mathtext}#1
\begin{tcolorbox}[colback=codebg]%
\color{code}%
\texttt{#2}%
\color{textcolor}%
\end{tcolorbox}\color{textcolor}}}

\def\backtick{`}
\def\curtime{\DTMdate{2023-09-21} ~\DTMtime{01:42:17}\DTMdisplayzone{-5}{00}}
\def\light#1{{\color{light}#1}}
\def\dark#1{{\color{dark}#1}}
\def\h{\hspace{1em}}

\def\signature{\href{mailto:marc@lange-data.org}{Dr. Marc Lange, marc@lange-data.org}}
\def\mailsignature{\href{mailto:marc@lange-data.org}{marc@lange-data.org}}

\def\abs#1{\left| #1 \right| }
\def\floor#1{\lfloor #1 \rfloor }

\newcommand{\lthree}{<\hspace{-5.2pt}3}
\newcommand{\E}{\mathcal{E}}
\newcommand{\DAG}{\mathcal{DAG}}
\newcommand{\Z}{\mathbb{Z}}
\newcommand{\R}{\mathbb{R}}
\renewcommand{\P}{\mathfrak{P}}
\newcommand{\Q}{\mathbb{Q}}
\newcommand{\N}{\mathbb{N}}
\renewcommand{\S}{\mathbb{S}}
\newcommand{\C}{\mathcal{C} }
\newcommand{\D}{\mathbb{D}}
\newcommand{\LL}{\mathcal{L}^1(\Z,\Z)}
\newcommand{\Set}{\mathrm{Set}}

\renewcommand{\dag}{\mathrm{emb}\mathcal{DAG}}


\newcommand{\s}{Set^{\phantom{\leq}}_{\phantom{+}}}
\newcommand{\spp}{Set=fSet}
\newcommand{\smp}{Set^{\leq}_{\phantom{+}}}
\newcommand{\spm}{Set^{\phantom{\leq}}_{+}}
\newcommand{\smm}{Set^{\leq}_{+}}
\newcommand{\spmpm}{Set^{\pm}_{\pm}}

\newcommand{\App}{A}
\newcommand{\Amp}{A^{\leq}_{\phantom{+}}}
\newcommand{\Apm}{A^{\phantom{\leq}}_{+}}
\newcommand{\Amm}{A^{\leq}_{+}}

\begin{document}
\chapter{The skeletal $2$-category of dags}
\section{The fundamental $2$-category}
In the spirit of module categories typically appearing in $K$-theory constructions, define
a $1$-category of directed acyclic graphs with very similar properties. Notice that a morphism
of directed acyclic graphs which is fixed on vertices can either exist or not, but there is
no freedom involved to allow two morphisms between that same source and target graph. The resulting
category is hence a literal $1$-category.
\defn{
    The skeletal category of finite embedded directed acyclic graphs, denoted $dag$ has
    as objects the natural numbers.

    The morphisms from $n$ to $n$ are directed acyclic graphs on the vertex set $\{1,\ldots,n\}$ which
    respect $\leq_\N$, i.e. an edge in a directed acyclic graph is only ever non-decreasing
    with respect to the natural order of natural numbers. Interpreted differently a morphism
    in $dag$ is a square matrix
    \[ A = (A_{ij}) 1\leq i,j \leq n, \]
    such that $A_{ii}=1$ and $A_{ij}=1 \Rightarrow 1\leq i\leq j\leq n$.
    I.e. it is upper triangular, and each diagonal element is set to $1$.

    Set composition to be $(B\cdot A)_{ij} = min(max(k|B_{ik}A_{kj}))$. I.e. set each entry to
    be defined as: If there is some middle $k$ connecting $i$ in $B$ to $k$, and then there
    is a $j$, such that $k$ is connected in $A$ to $j$, then $i,j$ is an edge in $B\cdot A$.
}

\prop{
    This is in fact a (small) category.
    \begin{proof}
    The existence of identities can be seen by setting $id_n$ to be the matrix with
    diagonal elements $1$ and rest $0$, then follows $A\cdot id = id \cdot A = A$ for any
    $A$ and $id$ of compatible degree.

    To check that $B\cdot A$ is well-defined, check if $B\cdot A$ in fact respects the
    $\leq_\N$ relation still. Given an edge $(i,j)\in B\cdot A$, there exists a vertex
    $k$, such that $(i,k)\in B$ and $(k,j)\in A$. Because both $A$ and $B$ are $\leq_\N$-compatibly
    directed, find $i\leq k$ and $k\leq j$, hence $i\leq j$, so $(i,j)$ is a legal edge
    in $\leq_\N$.

    For associativity note that the triple product $C\cdot B\cdot A$ has the natural "unbracketed"
    interpretation like above. It is a dag on the same vertex set as the three other graphs. It has
    edges the pairs of vertices $x,y$ such that $x$ has an edge to another vertex via $C$, that
    vertex has one to another in $B$ and finally that one has one via $A$ to $y$. There is nothing
    ineherently prioritised about the pairings, so it is associative.
    \end{proof}
}

\ex{
    Note in particular that multiplying a dag with itself $A\cdot A\cdot \ldots \cdot A$ when
    interpreted with $1$'s on the diagonal as above, eventually yields the transitive closure of $A$
    by the above reasoning.
}

\rem{
    Interpreting dag's as irreflexive, i.e. setting the diagonal to $0$'s should provide a descending
    filtration on $A$ until no edge remains. We have not investigated that route further so far.
}

\lem{
    For each pair of natural numbers $n,m\in\N$ consider the map:
    \[ \sigma_{n,m}\colon \lbrack n \rbrack \sqcup \lbrack m \rbrack \rightarrow \lbrack n+m \rbrack \]
    with $\sigma(k)=k$ for $k\in \lbrack n \rbrack$ and $\sigma(k)=k+n$ for $k\lbrack m \rbrack$.
    Then it has an inverse monotonic map of total orders with:
    \[\sigma^{-1}_{n,m}(k) = \begin{cases} i_n(k) & k\leq n\\ i_m(k-n)& n+1\leq k, \end{cases} \]
    where $i_?$ stands for the inclusion of the respective summand in the disjoint union.
    \begin{proof}
    One easily convinces themselves that $\sigma_{n,m}\circ \sigma^{-1}_{n,m} = id$ and
    $\sigma^{-1}_{n,m} \circ \sigma_{n,m} = id$.
    \end{proof}
}

\rem{
    The analogous product structure is far more interesting.
}

\prop{
    For each pair of natural numbers $n,m\in\N$ consider the map:
    \[ \omega_{n,m}\colon \lbrack n \rbrack \times \lbrack m \rbrack \rightarrow \lbrack nm \rbrack \]
    with $\omega_{n,m}(k,l) = (k-1)m+l$.
    It has inverse (not monotonic, since $\lbrack nm \rbrack$ has a fixed natural total order, the product
    has many to choose from):
    \[ \omega^{-1}_{n,m}(N) = ( N // m + 1, N mod m ), \]
    where $N // m$ is the integer part of the division of $N$ by $m$, e.g. $\lfloor N/m \rfloor$ provides
    a formula, and $N mod m$ has to be interpreted as the usual least natural number that can represent
    an element $mod m$, apart from $0$ which we set as $m$: $lm mod m = m$.
    \begin{proof}
    Note that the formulas are arranged just so that they invert each other, so there is not much to gain
    by displaying the actual calculation, which is quite short. The most insightful thing about them are the
    edge cases, convincing oneself that $0$ is never accidentally produced, $(1,1)$ goes to $1$
    and that $nm$ is actually hit.

    However, do note that both maps have
    the virtue of displaying their monotonicity in the formular: For $(a,b)\leq(c,d)$, i.e. $a\leq c$ and $b\leq d$,
    one can easily see $\omega(a,b)\leq \omega(c,d)$. Also the first component of the inverse map is
    (non-strictly) monotonically increasing, the second component, however, is not monotonic for $n\geq 2$, because
    it has to "fall" after each $m$-multiple in $\lbrack nm \rbrack$.
    \end{proof}
}

\cor{
    There are symmetries $\tau_\sigma$ and $\tau_\omega$ representing the $\sqcup$ symmetry and $\times$ symmetry on finite sets, as follows:
    \[\xymatrix{ \lbrack n \rbrack \sqcup \lbrack m \rbrack \ar[r]^\omega \ar[d]^{\tau_\sqcup} & \lbrack n+m \rbrack \ar[d]^{\tau_\sigma} \\
                 \lbrack m \rbrack \sqcup \lbrack n \rbrack \ar[r]^\omega & \lbrack m+n, \rbrack
    }\]
    and
    \[\xymatrix{ \lbrack n \rbrack \times \lbrack m \rbrack \ar[r]^\omega \ar[d]^{\tau_\times} & \lbrack nm \rbrack \ar[d]^{\tau_\omega} \\
                 \lbrack m \rbrack \times \lbrack n \rbrack \ar[r]^\omega & \lbrack mn \rbrack.
    }\]
    They each satisfy $\tau_\sigma^2=id$ and $\tau_\omega^2=id$, more explicitly for $\tau_\omega$, the two maps so defined are inverses of
    each other $\lbrack nm \rbrack \rightarrow \lbrack mn \rbrack \rightarrow \lbrack nm\rbrack$, where the objects are suggestively
    denoted to imply which $\tau_\omega$ has to be used. Similarly for $\tau_\sigma$.
    \begin{proof}
    Given the families $(\sigma_{n,m})$ and $(\omega_{n,m})$ as above
    \end{proof}
}


\prop{
    For $\lbrack n \rbrack = \{1,\ldots,n\}$ with $\lbrack 0 \rbrack = \emptyset$, consider the following
    bijections:
    The sum-bijection for each pair $n,m\in \N$:
    \[ \sigma_{n,m}\colon \lbrack n \rbrack \sqcup \lbrack m \rbrack \rightarrow \lbrack n+m \rbrack \]
    with $\sigma(k)=k$ for $k\in \lbrack n \rbrack$ and $\sigma(k)=k+n$ for $k\lbrack m \rbrack$.
    The product-bijection for each pair $n,m\in\N$:
    \[ \omega_{n,m}\colon \lbrack n \rbrack \times \lbrack m \rbrack \rightarrow \lbrack nm \rbrack \]
    with $\omega_{n,m}(k,l) = (k-1)m+l$.
    Both are monotonic bijections with respect to the natural orders on sum, product and the
    relevant subsets of natural numbers.

    They satisfy $\forall k,l,m,n\in \N$:
    \begin{enumerate}
    \item  \[\xymatrix{ \lbrack k \rbrack \sqcup \lbrack l \rbrack \sqcup \lbrack m \rbrack  \ar[d]^{\sigma_{k,l}\sqcup id_m}\ar[r]^{id_k\sqcup \sigma_{l,m}}
                      & \lbrack k \rbrack \sqcup \lbrack l+m \rbrack \ar[d]^{\sigma_{k,l+m}} \\
                        \lbrack k+l \rbrack \sqcup \lbrack m \rbrack \ar[r]^{\sigma_{k+l,m}}
                      & \lbrack k+l+m, \rbrack }\]
                      i.e. the family $\sigma_{\bullet,\bullet}$ induces a unique way to identify finite sums of
                      finite total orders as one finite total order, independent of how the summands are associated.

    \end{enumerate}

    then? (relations listen, bip folgern fuer dag, H ausrechnen, fertig)
}
\section{The fundamental bipermutative two element category}
\section{DAGs are a module category}
\end{document}