% This file is part of beamerthemepureminimalistic.

% If problems/bugs are found or enhancements are desired, please contact
% me over: https://github.com/kai-tub/latex-beamer-pure-minimalistic

\documentclass[a4paper]{book}
\include{version}
\usepackage{amsmath,amssymb,amsthm}
\newtheorem{thm}{Theorem}[section]
\newtheorem{theorem}{Theorem}[section]
\newtheorem{lem}[thm]{Lemma}
\newtheorem{lemma}[thm]{Lemma}
\newtheorem{prop}[thm]{Proposition}
\newtheorem{proposition}[thm]{Proposition}
\newtheorem{cor}[thm]{Corollary}
\newtheorem{corollary}[thm]{Corollary}
\theoremstyle{definition}
\newtheorem{Remi}[thm]{Reminder}
\newtheorem{rem}[thm]{Remark}
\newtheorem{defn}[thm]{Definition}
\newtheorem{ex}[thm]{Example}
\newtheorem{nonex}[thm]{{\bf NON}-Example}
\newtheorem{conj}[thm]{Conjecture}
\newtheorem{Exercise}[thm]{Exercise}
\newtheorem{Code}[thm]{Pseudocode}
\usepackage{datetime2}
\usepackage{graphicx}
\usepackage{svg}
% \svgpath{{/home/vassago/LATEX/}{/home/vassago/LATEX/svgs/}}
\usepackage{svg-extract}
\usepackage{hyperref}
\usepackage{inconsolata}

\usepackage{tcolorbox}

\usepackage[all]{xy}
\usepackage{xypic}
\usepackage{xcolor}

\definecolor{light}{RGB}{0,100,43}
\definecolor{dark}{RGB}{130,0,100}
\definecolor{exercise}{RGB}{220,200,255}
\definecolor{textcolor}{RGB}{0, 150, 128}
\definecolor{title}{RGB}{150, 0, 128}
\definecolor{footercolor}{RGB}{0, 128, 64}
\definecolor{code}{RGB}{240,220,160}
\definecolor{codebg}{RGB}{50,50,50}
\definecolor{mathbg}{RGB}{0,10,30}
\definecolor{mathtext}{RGB}{220,220,250}
\definecolor{bg}{RGB}{32, 0, 16}

\def\exercise#1{\color{light}{\h\Exercise{\color{exercise}\scshape #1}}\color{textcolor}}
\def\code#1#2{\color{dark}{\h\Code{\h}\\[.5em]%
\color{mathtext}#1
\begin{tcolorbox}[colback=codebg]%
\color{code}%
\texttt{#2}%
\color{textcolor}%
\end{tcolorbox}\color{textcolor}}}

\def\backtick{`}
\def\curtime{\DTMdate{2023-09-21} ~\DTMtime{01:42:17}\DTMdisplayzone{-5}{00}}
\def\light#1{{\color{light}#1}}
\def\dark#1{{\color{dark}#1}}
\def\h{\hspace{1em}}

\def\signature{\href{mailto:marc@lange-data.org}{Dr. Marc Lange, marc@lange-data.org}}
\def\mailsignature{\href{mailto:marc@lange-data.org}{marc@lange-data.org}}

\def\abs#1{\left| #1 \right| }
\def\floor#1{\lfloor #1 \rfloor }

\newcommand{\lthree}{<\hspace{-5.2pt}3}
\newcommand{\E}{\mathcal{E}}
\newcommand{\DAG}{\mathcal{DAG}}
\newcommand{\Z}{\mathbb{Z}}
\newcommand{\R}{\mathbb{R}}
\newcommand{\C}{\mathbb{C}}
\renewcommand{\H}{\mathbb{H}}
\renewcommand{\k}{\mathbb{k}}
\renewcommand{\P}{\mathfrak{P}}
\newcommand{\Q}{\mathbb{Q}}
\newcommand{\N}{\mathbb{N}}
\renewcommand{\S}{\mathbb{S}}
\newcommand{\D}{\mathbb{D}}
\newcommand{\LL}{\mathcal{L}^1(\Z,\Z)}
\newcommand{\Set}{\mathrm{Set}}

\renewcommand{\dag}{\mathrm{emb}\mathcal{DAG}}


\newcommand{\s}{Set^{\phantom{\leq}}_{\phantom{+}}}
\newcommand{\spp}{Set=fSet}
\newcommand{\smp}{Set^{\leq}_{\phantom{+}}}
\newcommand{\spm}{Set^{\phantom{\leq}}_{+}}
\newcommand{\smm}{Set^{\leq}_{+}}
\newcommand{\spmpm}{Set^{\pm}_{\pm}}

\newcommand{\App}{A}
\newcommand{\Amp}{A^{\leq}_{\phantom{+}}}
\newcommand{\Apm}{A^{\phantom{\leq}}_{+}}
\newcommand{\Amm}{A^{\leq}_{+}}

\begin{document}
\chapter{2024-04-21 -- Cyclotomic Polynomials}
\section{Explicit construction}
For the cyclotomic polynomials there is always
a coefficient which is exactly $1\in \Z$ and maps to the relevant $1\in R$
for any commutative unital zero-divisor-free factorial ring over which we consider the cyclotomic
polynomial. Hence recall the famous Eisenstein's criterion to look up in your favourite
algebra reference, with simplification to $\Z$ and $\Q$.
\prop{
    Let $f=\sum_ia_iX^i\in \Z\lbrack X \rbrack$ be a polynomial with coefficients in $\Z$ of degree $N$
    which is monic, i.e. a polynomial of degree $N$ such that $a_N = 1$.

    Assume $a_0=\pm p$ for $p\in \N$ a prime number, and assume in addition $p|a_i$ for each $a_i$ with
    $i=1,\ldots,N-1$. Then $f$ is irreducible in $\Z\lbrack X \rbrack$ and $\Q\lbrack X \rbrack$.
    \begin{proof}
    Let $f=gh$ in $\Q\lbrack X \rbrack$. In fact the factors can be chosen as $g,h\in\Z\lbrack X\rbrack$
    with both degrees strictly smaller than $f$'s.

    For $p\in\Z$ prime the ideal $(p)\subset \Z$ is a prime ideal, with quotient $\Z/(p) = \mathbb{F}_p$.
    On coefficients this induces a reduction ring homomorphism:
    \[ \pi\colon \Z\lbrack X\rbrack \rightarrow \Z/p\lbrack X\rbrack. \]
    By the assumptions on $f$ get $\pi(f) = X^N$, but also $\pi(g)\pi(h)=X^N$ because $f=gh$ by the
    assumption before. Since $\mathbb{F}_p\lbrack X \rbrack$ is a euclidean domain it is also factorial,
    so $\pi(g)=a_i X^i$ and $\pi(h) = b_j X^j$ such that $i+j=N$ and $a_ib_j=1\in\Z/p$.

    Hence we get for the integral $g,h$: $p|g_0$ and $p|h_0$, hence follows $p^2|a_0$, but we assumed
    $a_0=p$ prime, which is a contradiction. So $f$ was in fact irreducible in $\Z \lbrack X\rbrack$
    and $\Q \lbrack X\rbrack$.
    \end{proof}
}

\end{document}