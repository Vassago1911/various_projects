% This file is part of beamerthemepureminimalistic.

% If problems/bugs are found or enhancements are desired, please contact
% me over: https://github.com/kai-tub/latex-beamer-pure-minimalistic

\documentclass[a4paper]{book}
\include{version}
\usepackage{amsmath,amssymb,amsthm}
\newtheorem{thm}{Theorem}[section]
\newtheorem{theorem}{Theorem}[section]
\newtheorem{lem}[thm]{Lemma}
\newtheorem{lemma}[thm]{Lemma}
\newtheorem{prop}[thm]{Proposition}
\newtheorem{proposition}[thm]{Proposition}
\newtheorem{cor}[thm]{Corollary}
\newtheorem{corollary}[thm]{Corollary}
\theoremstyle{definition}
\newtheorem{Remi}[thm]{Reminder}
\newtheorem{rem}[thm]{Remark}
\newtheorem{defn}[thm]{Definition}
\newtheorem{ex}[thm]{Example}
\newtheorem{nonex}[thm]{{\bf NON}-Example}
\newtheorem{conj}[thm]{Conjecture}
\newtheorem{Exercise}[thm]{Exercise}
\newtheorem{Code}[thm]{Pseudocode}
\usepackage{datetime2}
\usepackage{graphicx}
\usepackage{svg}
% \svgpath{{/home/vassago/LATEX/}{/home/vassago/LATEX/svgs/}}
\usepackage{svg-extract}
\usepackage{stmaryrd}
\usepackage{hyperref}
\usepackage{inconsolata}

\usepackage{tcolorbox}

\usepackage[all]{xy}
\usepackage{xypic}
\usepackage{xcolor}

\definecolor{light}{RGB}{0,100,43}
\definecolor{dark}{RGB}{130,0,100}
\definecolor{exercise}{RGB}{220,200,255}
\definecolor{textcolor}{RGB}{0, 150, 128}
\definecolor{title}{RGB}{150, 0, 128}
\definecolor{footercolor}{RGB}{0, 128, 64}
\definecolor{code}{RGB}{240,220,160}
\definecolor{codebg}{RGB}{50,50,50}
\definecolor{mathbg}{RGB}{0,10,30}
\definecolor{mathtext}{RGB}{220,220,250}
\definecolor{bg}{RGB}{32, 0, 16}

\def\exercise#1{\color{light}{\h\Exercise{\color{exercise}\scshape #1}}\color{textcolor}}
\def\code#1#2{\color{dark}{\h\Code{\h}\\[.5em]%
\color{mathtext}#1
\begin{tcolorbox}[colback=codebg]%
\color{code}%
\texttt{#2}%
\color{textcolor}%
\end{tcolorbox}\color{textcolor}}}

\def\backtick{`}
\def\curtime{\DTMdate{2023-09-21} ~\DTMtime{01:42:17}\DTMdisplayzone{-5}{00}}
\def\light#1{{\color{light}#1}}
\def\dark#1{{\color{dark}#1}}
\def\h{\hspace{1em}}

\def\signature{\href{mailto:marc@lange-data.org}{Dr. Marc Lange, marc@lange-data.org}}
\def\mailsignature{\href{mailto:marc@lange-data.org}{marc@lange-data.org}}

\def\abs#1{\left| #1 \right| }
\def\floor#1{\lfloor #1 \rfloor }

\newcommand{\lthree}{<\hspace{-5.2pt}3}
\newcommand{\E}{\mathcal{E}}
\newcommand{\DAG}{\mathcal{DAG}}
\newcommand{\Z}{\mathbb{Z}}
\newcommand{\R}{\mathbb{R}}
\newcommand{\C}{\mathbb{C}}
\renewcommand{\H}{\mathbb{H}}
\renewcommand{\k}{\mathbb{k}}
\renewcommand{\P}{\mathfrak{P}}
\newcommand{\Q}{\mathbb{Q}}
\newcommand{\N}{\mathbb{N}}
\renewcommand{\S}{\mathbb{S}}
\newcommand{\D}{\mathbb{D}}
\newcommand{\LL}{\mathcal{L}^1(\Z,\Z)}
\newcommand{\Set}{\mathrm{Set}}

\renewcommand{\dag}{\mathrm{emb}\mathcal{DAG}}


\newcommand{\s}{Set^{\phantom{\leq}}_{\phantom{+}}}
\newcommand{\spp}{Set=fSet}
\newcommand{\smp}{Set^{\leq}_{\phantom{+}}}
\newcommand{\spm}{Set^{\phantom{\leq}}_{+}}
\newcommand{\smm}{Set^{\leq}_{+}}
\newcommand{\spmpm}{Set^{\pm}_{\pm}}

\newcommand{\App}{A}
\newcommand{\Amp}{A^{\leq}_{\phantom{+}}}
\newcommand{\Apm}{A^{\phantom{\leq}}_{+}}
\newcommand{\Amm}{A^{\leq}_{+}}

\begin{document}
\chapter{2024-04-21 -- Cyclotomic Polynomials}
\section{Polynomials and power series with integer coefficients}
\defn{
    Consider the set $\Z\llbracket X \rrbracket = \{~\sum_{i\geq0}c_iX^i~|~c_i\in\Z~\}$
    of not necessarily finite sums in the monomials $X^i$ with multiplication defined
    by $(f(X)g(X))_i = \sum_k f_kg_{i-k}$ for each coefficient index $i\geq0$. It is a
    commutative ring with unit $1\in\Z \subset \Z\llbracket X \rrbracket$.

    It naturally includes the ring of integral polynomials $\Z\lbrack X \rbrack\subset\Z\llbracket X \rrbracket$,
    which are integral power series with only finitely many non-zero coefficients and the multiplication
    as induced from power series. In particular $1\in \Z \subset \Z\lbrack X \rbrack\subset\Z\llbracket X \rrbracket$
    each have the same unit considered along the canonical subset inclusions.
}

It is confusing to assemble all the facts about units and primes $\Z \subset \Z\lbrack X \rbrack\subset\Z\llbracket X \rrbracket$
from the literature, so I summarise and prove them here as far as elementarily possible and conveniently enlightening.

\prop{
    The, multiplicative invertible elements, in short: units of $\Z$ are plain the signs $\Z^\times = \{\pm 1\}$.
    The units in integral power series are given by $\Z\llbracket X \rrbracket^\times = \{~\sum_{i\geq0}c_iX^i~|~c_i\in\Z~\wedge~c_0\in\{\pm1\}~\}$.
    Since each non-constant power series which is a unit in power series is either not a polynomial itself, or its inverse is a properly
    infinite power series, get $\Z\lbrack X\rbrack^\times = \Z^\times = \{\pm1\}$.
    \begin{proof}
    Clearly the units in $\Z$ are exactly the non-zero elements $n$, which can be inverted in $\Z$, i.e. where $\frac1n \in \Q\cap\Z$.
    So $|n|\geq2$ is clearly not invertible in $\Z$, $0$ is not invertible anywhere, but $-1,1$ clearly are each their own multiplicative
    inverse in any unital ring.

    Let $p = \sum c_i X^i$ be an integral power series which is a unit, i.e. for which there is a unique $q = \sum d_i X^i$, such that:
    \[ p\cdot q = \sum_{i,j} c_i\cdot d_j X^{i+j} = 1.\]
    In degree $0$ it follows:
    \[ c_0d_0 = 1, \]
    since the units of $\Z$ are $\{\pm1\}$ without loss of generality we can assume $c_0=d_0=1$ by multiplying $p,q$ each by $-1$. In particular
    the subset inclusion follows \[\Z\llbracket X \rrbracket^\times \subset \{~\sum_{i\geq0}c_iX^i~|~c_i\in\Z~\wedge~c_0\in\{\pm1\}~\}\].

    For the $\supset$-inclusion consider without loss of generality a $p = 1+\sum_{i\geq1}c_iX^i$ by multiplying with $-1$ if necessary.
    As above we find necessarily an inverse power series has to start with the same constant term $d_0 = 1$. Hence we get in degree $1$:
    \[ d_1 = -c_1. \]
    It follows in degree $2$:
    \[ c_0d_2 + c_1d_1 + c_2d_0 = 0 \]
    giving
    \[ d_2 = c_1^2 - c_2. \]

    Inductively assume $d_i$ determined up to $n-1$ and consider degree $n$:
    \[ 0 = \sum_{i=0,\ldots,n}c_id_{n-i} = d_n + \sum_{i=0,\ldots,n-1}c_id_{n-i} \]
    which gives
    \[ d_n = -\sum_{i=0,\ldots,n-1}c_id_{n-i}. \]

    Then $q(X)=\sum_n d_nX^n$ satisfies $pq=1$ by the inductive construction of its coefficients, so $p$ is a unit in integral power series.

    Finally consider $p\in \Z\lbrack X \rbrack^\times$. On non-zero polynomials with integral coefficients we have a well-defined degree, i.e.
    a map $\nu\colon \Z\lbrack X \rbrack \rightarrow \mathbb{N}$ which satisfies $\nu(f\cdot g)=\nu(f)+\nu(g)$, given by assigning to each
    polynomial the highest index such that its coefficient is non-zero.

    In particular it follows for $k\in \Z$ and $p\in \Z\lbrack X\rbrack\setminus\{0\}$ arbitary, $0=deg(k)$ and $deg(kf)=deg(f)>0$.
    For a unit we hence get $0 = deg(1) = deg(pq) = deg(p)+deg(q)$, hence follows $deg(p)=deg(q)=0$ in natural numbers, so
    $p \in \{\pm1\}$ with no non-trivial higher terms. If $p$ is an integral polynomial invertible considered as a power series,
    then follows "$deg(q)=\infty$". I.e. if there were a highest non-trivial coefficient for $q$, then $pq=1$ forces $p$ and $q$
    to be constant and in the units of $\Z$.
    \end{proof}
}

\rem{
    Do note how that describes the units in the integral power series ring which are themselves polynomials. Since there
    is not a well-defined degree map like on polynomials anymore, a $q$ inverting a polynomial $p$ multiplicatively
    can escape to "infinite degree", i.e. the inductive process describing the $d_n$ does not stop to produce non-trivial coefficients.
    Thus no non-constant polynomial can have a multiplicative inverse in integral polynomials.
}

\prop{
    Let $f=gh$ in integral polynomials with $f=\sum_i f_i X^i,g=\sum_i g_i X^i,h=\sum_i h_i X^i$ each finite sums with integer
    coefficients. Assume $f_0=1$ and without loss of generality $g_0=h_0=1$. It follows $f,g,h\in \Z\llbracket X \rrbracket^\times$, and
    $f^{-1}=h^{-1}g^{-1}$ with each factor a properly infinite power series, each having constant term $1$ as well.
}

\cor{
    The units of integral power series decompose as \[\Z\llbracket X\rrbracket^\times \cong \{\pm1\}\oplus X\Z\llbracket X\rrbracket. \]
    \begin{proof}
    Let $p$ be a unit in integral power series, i.e. $p = \pm1 + \sum_{i\geq1} c_iX^i = \pm1 + X\sum_ic_iX^{i-1}$ by our proposition above.
    The decomposition as indicated defines a map into the product, which is evidently injective and surjective. It is also just regarding
    a formal sum as a sum in a polynomial ring for the inverse map, one could regard the isomorphism as formal nonsense.
    \end{proof}
}

\cor{
    For a unit $f\in\Z\llbracket X \rrbracket^\times$ get its inverse $g$, then $f(X^n)$ has the same constant term as
    $f$ does, and $g(X^n)$ is the inverse for $f(X^n)$.
    \begin{proof}
    Consider $fg=1$ as $f(X)g(X) = 1$, then follows $f(X^n)g(X^n)=1$, hence the claim.
    \end{proof}
}


\section{Explicit construction}
\defn{
Call a polynomial $f=\sum_ia_iX^i\in \Z\lbrack X \rbrack$ with $a_0=1$ irreducible,
if $f = gh$ for $g,h\in \Z\lbrack X \rbrack$ and $g\not\in\{\pm1\}$ implies $h\in\{\pm1\}$.
}

\prop{
A polynomial $f=\sum_ia_iX^i\in \Z\lbrack X \rbrack$ is irreducible if
and only if any and hence all of its translates $f_z(X):=f(X-z) ~~z\in\Z$
are irreducible.
\begin{proof}
If $f_z(X)$ were decomposable non-trivially as $f_z(X)=g(X)h(X)$, then get
$f(X-z)=g(X)h(X)$, hence $f(X)=g(X+z)h(X+z)$ decomposes $f$.
\end{proof}
}

\prop{
If $f=\sum_ia_iX^i$ is irreducible with $a_0=1$, then so is each of the polynomials given
by inserting a power of $X$: $f_n(X) := f(X^n)$ with $n\geq2$.
\begin{proof}
Assume we had a decomposition in $\Z\lbrack X\rbrack$ of $f_n$:
$\sum_ia_iX^{ni}=f(X^n) = f_n(X) = g(X)h(X).$
Show that $g,h$ each area also of the form $\bar g(X^n)$ and $\bar h(X^n)$, hence
$Z = X^n$ gives a decomposition $f(Z)=g(Z)h(Z)$.

Assume to contradiction for $g$ and then necessarily $h$ a coefficient $g_i$ and $h_{kn-i}$ both
not equal to zero and $g_i$ the $i$-minimal coefficient in $g$, such that $i$ is not a multiple
of $n$.

By multiplying $g,h$ each with a sign, we can assume $1=a_0 = 1\cdot 1 = g_0\cdot h_0$ with
$g_0=h_0=1$. It follows $g = 1 + g_i X^i + \sum_{j>i}g_j X^j$ and $h = 1 + \sum_{j\geq1}h_jX^j$.

\end{proof}
}

For the cyclotomic polynomials there is always
a coefficient which is exactly $1\in \Z$ and maps to the relevant $1\in R$
for any commutative unital zero-divisor-free factorial ring over which we consider the cyclotomic
polynomial. Hence recall the famous Eisenstein's criterion to look up in your favourite
algebra reference, with simplification to $\Z$ and $\Q$.
\prop{
    Let $f=\sum_ia_iX^i\in \Z\lbrack X \rbrack$ be a polynomial with coefficients in $\Z$ of degree $N$
    which is monic, i.e. a polynomial of degree $N$ such that $a_N = 1$.

    Assume $a_0=\pm p$ for $p\in \N$ a prime number, and assume in addition $p|a_i$ for each $a_i$ with
    $i=1,\ldots,N-1$. Then $f$ is irreducible in $\Z\lbrack X \rbrack$ and $\Q\lbrack X \rbrack$.
    \begin{proof}
    Let $f=gh$ in $\Q\lbrack X \rbrack$. In fact the factors can be chosen as $g,h\in\Z\lbrack X\rbrack$
    with both degrees strictly smaller than $f$'s.

    For $p\in\Z$ prime the ideal $(p)\subset \Z$ is a prime ideal, with quotient $\Z/(p) = \mathbb{F}_p$.
    On coefficients this induces a reduction ring homomorphism:
    \[ \pi\colon \Z\lbrack X\rbrack \rightarrow \Z/p\lbrack X\rbrack. \]
    By the assumptions on $f$ get $\pi(f) = X^N$, but also $\pi(g)\pi(h)=X^N$ because $f=gh$ by the
    assumption before. Since $\mathbb{F}_p\lbrack X \rbrack$ is a euclidean domain it is also factorial,
    so $\pi(g)=a_i X^i$ and $\pi(h) = b_j X^j$ such that $i+j=N$ and $a_ib_j=1\in\Z/p$.

    Hence we get for the integral $g,h$: $p|g_0$ and $p|h_0$, hence follows $p^2|a_0$, but we assumed
    $a_0=p$ prime, which is a contradiction. So $f$ was in fact irreducible in $\Z \lbrack X\rbrack$
    and $\Q \lbrack X\rbrack$.
    \end{proof}
}

\end{document}