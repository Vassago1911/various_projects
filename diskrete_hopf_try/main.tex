% This file is part of beamerthemepureminimalistic.

% If problems/bugs are found or enhancements are desired, please contact
% me over: https://github.com/kai-tub/latex-beamer-pure-minimalistic

\documentclass[a4paper]{book}
\include{version}
\usepackage{amsmath,amssymb,amsthm}
\newtheorem{thm}{Theorem}[section]
\newtheorem{theorem}{Theorem}[section]
\newtheorem{lem}[thm]{Lemma}
\newtheorem{lemma}[thm]{Lemma}
\newtheorem{prop}[thm]{Proposition}
\newtheorem{proposition}[thm]{Proposition}
\newtheorem{cor}[thm]{Corollary}
\newtheorem{corollary}[thm]{Corollary}
\theoremstyle{definition}
\newtheorem{Remi}[thm]{Reminder}
\newtheorem{rem}[thm]{Remark}
\newtheorem{defn}[thm]{Definition}
\newtheorem{ex}[thm]{Example}
\newtheorem{nonex}[thm]{{\bf NON}-Example}
\newtheorem{conj}[thm]{Conjecture}
\newtheorem{Exercise}[thm]{Exercise}
\newtheorem{Code}[thm]{Pseudocode}
\usepackage{datetime2}
\usepackage{graphicx}
\usepackage{svg}
% \svgpath{{/home/vassago/LATEX/}{/home/vassago/LATEX/svgs/}}
\usepackage{svg-extract}
\usepackage{hyperref}
\usepackage{inconsolata}

\usepackage{tcolorbox}

\usepackage[all]{xy}
\usepackage{xypic}
\usepackage{xcolor}

\definecolor{light}{RGB}{0,100,43}
\definecolor{dark}{RGB}{130,0,100}
\definecolor{exercise}{RGB}{220,200,255}
\definecolor{textcolor}{RGB}{0, 150, 128}
\definecolor{title}{RGB}{150, 0, 128}
\definecolor{footercolor}{RGB}{0, 128, 64}
\definecolor{code}{RGB}{240,220,160}
\definecolor{codebg}{RGB}{50,50,50}
\definecolor{mathbg}{RGB}{0,10,30}
\definecolor{mathtext}{RGB}{220,220,250}
\definecolor{bg}{RGB}{32, 0, 16}

\def\exercise#1{\color{light}{\h\Exercise{\color{exercise}\scshape #1}}\color{textcolor}}
\def\code#1#2{\color{dark}{\h\Code{\h}\\[.5em]%
\color{mathtext}#1
\begin{tcolorbox}[colback=codebg]%
\color{code}%
\texttt{#2}%
\color{textcolor}%
\end{tcolorbox}\color{textcolor}}}

\def\backtick{`}
\def\curtime{\DTMdate{2023-09-21} ~\DTMtime{01:42:17}\DTMdisplayzone{-5}{00}}
\def\light#1{{\color{light}#1}}
\def\dark#1{{\color{dark}#1}}
\def\h{\hspace{1em}}

\def\signature{\href{mailto:marc@lange-data.org}{Dr. Marc Lange, marc@lange-data.org}}
\def\mailsignature{\href{mailto:marc@lange-data.org}{marc@lange-data.org}}

\def\abs#1{\left| #1 \right| }
\def\floor#1{\lfloor #1 \rfloor }

\newcommand{\lthree}{<\hspace{-5.2pt}3}
\newcommand{\E}{\mathcal{E}}
\newcommand{\DAG}{\mathcal{DAG}}
\newcommand{\Z}{\mathbb{Z}}
\newcommand{\R}{\mathbb{R}}
\newcommand{\C}{\mathbb{C}}
\renewcommand{\H}{\mathbb{H}}
\renewcommand{\k}{\mathbb{k}}
\renewcommand{\P}{\mathfrak{P}}
\newcommand{\Q}{\mathbb{Q}}
\newcommand{\N}{\mathbb{N}}
\renewcommand{\S}{\mathbb{S}}
\newcommand{\D}{\mathbb{D}}
\newcommand{\LL}{\mathcal{L}^1(\Z,\Z)}
\newcommand{\Set}{\mathrm{Set}}

\renewcommand{\dag}{\mathrm{emb}\mathcal{DAG}}


\newcommand{\s}{Set^{\phantom{\leq}}_{\phantom{+}}}
\newcommand{\spp}{Set=fSet}
\newcommand{\smp}{Set^{\leq}_{\phantom{+}}}
\newcommand{\spm}{Set^{\phantom{\leq}}_{+}}
\newcommand{\smm}{Set^{\leq}_{+}}
\newcommand{\spmpm}{Set^{\pm}_{\pm}}

\newcommand{\App}{A}
\newcommand{\Amp}{A^{\leq}_{\phantom{+}}}
\newcommand{\Apm}{A^{\phantom{\leq}}_{+}}
\newcommand{\Amm}{A^{\leq}_{+}}

\begin{document}
\chapter{2024-04-20 -- The basic Hopf fibre bundles}
\section{Explicit construction}
\defn{
    For $k=\R^n$ assume a multiplication $\mu\colon k\otimes_\R k \rightarrow k$
    making $k^\times \supset \S(k) \cong \S^{n-1}$ into an $H$-space. Then we can
    construct a fibre bundle as follow: Consider the unit sphere in $\S(k\times k)$,
    which can be identified as $\S(\R^{2n})=\S^{2n-1}$, and the one-point compactification
    of $\S^k = \S^{dim_\R k} = \S^n$.

    Notice that for $\S(k^2)$ the components can be written as pairs of elements
    of $k$ with norms each less than or equal to one, and norm-squares summing to one.
    Thus consider the map:
    \[ \begin{aligned}
    m_k &\colon & \S^{2n-1} = \S(k\times k) &\rightarrow & \S^k \\
         &       & a,b                        &\mapsto     & a\cdot b^{-1},
    \end{aligned} \]
    where $(\bullet) \cdot b^{-1}$ has to be defined as $\infty$ for $b=0$, (note that
    this implies $|a|=1$, so $a\neq 0$.)
}

\prop{
    This is a fibre bundle for $k=\R,\C,\H$, with fibre $F = \S(k)=\S^0,\S^1,\S^3$.
    \begin{proof}
    Consider first the case $k=\R$.
    The case $k=\R$ can be identified as $\Z/2 \rightarrow \S^1 \rightarrow \S^1$, where the first circle
    is a unit sphere in $\R^2$ and the second circle is the one-point-compactification of $k=\R$.
    Let $t \in \R^+$ with $|t|< 1$, it defines a subset of $\S(\R\times\R)$ with
    \[U_t:=\{~(x,y)\in \S(\R\times\R)~|~xy^{-1}=t\wedge |x|^2+|y|^2=1~\}.\]
    With $|t|<1$ and $z,t\in\R$ the fibre condition and the unit sphere condition assemble to give:
    \[
    \begin{aligned}
    xy^{-1}&=t\\
    |x|^2+|y|^2&=1\\
    |t|&\leq 1
    \end{aligned}
    \]
    \end{proof}
}

\end{document}