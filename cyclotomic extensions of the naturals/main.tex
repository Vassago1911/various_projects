% This file is part of beamerthemepureminimalistic.

% If problems/bugs are found or enhancements are desired, please contact
% me over: https://github.com/kai-tub/latex-beamer-pure-minimalistic

\documentclass[a4paper]{book}
\include{version}
\usepackage{amsmath,amssymb,amsthm}
\newtheorem{thm}{Theorem}[section]
\newtheorem{theorem}{Theorem}[section]
\newtheorem{lem}[thm]{Lemma}
\newtheorem{lemma}[thm]{Lemma}
\newtheorem{prop}[thm]{Proposition}
\newtheorem{proposition}[thm]{Proposition}
\newtheorem{cor}[thm]{Corollary}
\newtheorem{corollary}[thm]{Corollary}
\theoremstyle{definition}
\newtheorem{Remi}[thm]{Reminder}
\newtheorem{rem}[thm]{Remark}
\newtheorem{defn}[thm]{Definition}
\newtheorem{ex}[thm]{Example}
\newtheorem{nonex}[thm]{{\bf NON}-Example}
\newtheorem{conj}[thm]{Conjecture}
\newtheorem{Exercise}[thm]{Exercise}
\newtheorem{Code}[thm]{Pseudocode}
\usepackage{datetime2}
\usepackage{graphicx}
\usepackage{svg}
% \svgpath{{/home/vassago/LATEX/}{/home/vassago/LATEX/svgs/}}
\usepackage{svg-extract}
\usepackage{stmaryrd}
\usepackage{hyperref}
\usepackage{inconsolata}

\usepackage{tcolorbox}

\usepackage[all]{xy}
\usepackage{xypic}
\usepackage{xcolor}

\definecolor{light}{RGB}{0,100,43}
\definecolor{dark}{RGB}{130,0,100}
\definecolor{exercise}{RGB}{220,200,255}
\definecolor{textcolor}{RGB}{0, 150, 128}
\definecolor{title}{RGB}{150, 0, 128}
\definecolor{footercolor}{RGB}{0, 128, 64}
\definecolor{code}{RGB}{240,220,160}
\definecolor{codebg}{RGB}{50,50,50}
\definecolor{mathbg}{RGB}{0,10,30}
\definecolor{mathtext}{RGB}{220,220,250}
\definecolor{bg}{RGB}{32, 0, 16}

\def\exercise#1{\color{light}{\h\Exercise{\color{exercise}\scshape #1}}\color{textcolor}}
\def\code#1#2{\color{dark}{\h\Code{\h}\\[.5em]%
\color{mathtext}#1
\begin{tcolorbox}[colback=codebg]%
\color{code}%
\texttt{#2}%
\color{textcolor}%
\end{tcolorbox}\color{textcolor}}}

\def\backtick{`}
\def\curtime{\DTMdate{2023-09-21} ~\DTMtime{01:42:17}\DTMdisplayzone{-5}{00}}
\def\light#1{{\color{light}#1}}
\def\dark#1{{\color{dark}#1}}
\def\h{\hspace{1em}}

\def\signature{\href{mailto:marc@lange-data.org}{Dr. Marc Lange, marc@lange-data.org}}
\def\mailsignature{\href{mailto:marc@lange-data.org}{marc@lange-data.org}}

\def\abs#1{\left| #1 \right| }
\def\floor#1{\lfloor #1 \rfloor }

\newcommand{\lthree}{<\hspace{-5.2pt}3}
\newcommand{\E}{\mathcal{E}}
\newcommand{\DAG}{\mathcal{DAG}}
\newcommand{\Z}{\mathbb{Z}}
\newcommand{\R}{\mathbb{R}}
\newcommand{\C}{\mathbb{C}}
\renewcommand{\H}{\mathbb{H}}
\renewcommand{\k}{\mathbb{k}}
\renewcommand{\P}{\mathfrak{P}}
\newcommand{\Q}{\mathbb{Q}}
\newcommand{\N}{\mathbb{N}}
\renewcommand{\S}{\mathbb{S}}
\newcommand{\D}{\mathbb{D}}
\newcommand{\LL}{\mathcal{L}^1(\Z,\Z)}
\newcommand{\Set}{\mathrm{Set}}

\renewcommand{\dag}{\mathrm{emb}\mathcal{DAG}}


\newcommand{\s}{Set^{\phantom{\leq}}_{\phantom{+}}}
\newcommand{\spp}{Set=fSet}
\newcommand{\smp}{Set^{\leq}_{\phantom{+}}}
\newcommand{\spm}{Set^{\phantom{\leq}}_{+}}
\newcommand{\smm}{Set^{\leq}_{+}}
\newcommand{\spmpm}{Set^{\pm}_{\pm}}

\newcommand{\App}{A}
\newcommand{\Amp}{A^{\leq}_{\phantom{+}}}
\newcommand{\Apm}{A^{\phantom{\leq}}_{+}}
\newcommand{\Amm}{A^{\leq}_{+}}

\begin{document}
\defn{
    Consider the category of commutative unital rigs $Crig$. An object is a
    crig, that is "ring without negatives", i.e. a tuple $(A,+,\cdot,1)$
    with $A$ a set, maps $+\colon A\times A\rightarrow A$, $\cdot\colon A\times A\rightarrow A$,
    which are associative and distributive in the usual way as for rings.
    However $(A,+)$ is only required to be a commutative composition, an additive neutral element is not necessarily required,
    and $(A,\cdot,1)$ is a commutative monoid with neutral element $1$ as is the case for rings, too.

    A morphism of unital crigs $\varphi\colon A \rightarrow B$ is a set map $A\rightarrow B$ with $\varphi(1)=1$,
    which is a monoid morphism with respect to $+$ and $\cdot$.
}

\rem{
    Since the existence of negatives from the onset of a problem is the exception
    rather than the rule in this text, we shall rarely refer to groups as such. Specifically
    only commutative groups feature as completions or other monoid quotients which just
    happen to have additive inverses.
}

\ex{
    THE example of study in this text are the natural numbers $\mathbb{N}$. Note that
    the definition left a tiny amount of wiggle room by not requiring a zero element, hence
    the set of positive natural numbers $\mathbb{N}_{>0}=\{1,\ldots,n,n+1,\ldots\}$ has
    a unique natural crig morphism to every other crig. It is given by unitality
    and induction over the fact that each natural number greater or equal to $2$
    can be written as a sum of two smaller numbers. In particular the natural numbers
    including zero $\mathbb{N}=\mathbb{N_0}$ have a unique crig morphism $\mathbb{N}_{>0} \rightarrow \mathbb{N}_0$,
    but there is no crig map in the reverse direction: Any image of zero $f0$ would have to
    satisfy $f(0)+f(0) = f(0+0) = f(0)$, but this is not possible in positive natural numbers.
}

The specific argument identifies a useful property the natural numbers but many other crigs
enjoy too, cancellation.

\defn{
    An associative and commutative composition $(A,+)$ is said to {\bf satisfy cancellation}, if
    \[\forall a,b\in A\colon \exists k \in A\colon a+k=b+k \Rightarrow a=b.\]
    That is, no two elements share an translate by the same element.

    A commutative monoid $(A,+,0)$ with the cancellation property we call {\bf strongly non-negative},
    if \[ \forall a,b\in A\colon a+b=0 \Rightarrow a=b=0. \]
    That is, for each element in $A$ its negative is not already an element of A, unless the element is the zero element.
}

\rem{
    One can stumble into the strongle non-negative property in various ways. It could also be considered
    a very strong form of torsion-freeness. In case of $\mathbb{N}$ both perspectives coincide, so it
    does not matter for our study.

    I do not expect this property to be of much use beyond $\mathbb{N}$ and its polynomial crigs, but
    bipermutative rig categories with objects $\mathbb{N}$ come up easily in $K$-theory constructions,
    so that case is quite enough motivation to abstract away into an important property like this.
}

\prop{
    Each crig $A$ has an associated crig of polynomials in one variable $A\lbrack X\rbrack$, satisfying the
    usual universal property: There is a crig inclusion $A\rightarrow A\lbrack X\rbrack$,
    and for each crig morphism $\varphi\colon A\rightarrow B$ with target crig $B$
    and each target element $b\in B$ there exists a unique extension of $\varphi$ given
    by "evaluating $X$ at $b$":
    \[
    \xymatrix{
    A\ar[r]\ar[d]&B\ar[d]\\C\ar[r]&D.
    }
    \]
}

\end{document}