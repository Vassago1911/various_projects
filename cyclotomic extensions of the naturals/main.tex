% This file is part of beamerthemepureminimalistic.

% If problems/bugs are found or enhancements are desired, please contact
% me over: https://github.com/kai-tub/latex-beamer-pure-minimalistic

\documentclass[a4paper]{book}
\include{version}
\usepackage{amsmath,amssymb,amsthm}
\newtheorem{thm}{Theorem}[section]
\newtheorem{theorem}{Theorem}[section]
\newtheorem{lem}[thm]{Lemma}
\newtheorem{lemma}[thm]{Lemma}
\newtheorem{prop}[thm]{Proposition}
\newtheorem{proposition}[thm]{Proposition}
\newtheorem{cor}[thm]{Corollary}
\newtheorem{corollary}[thm]{Corollary}
\theoremstyle{definition}
\newtheorem{Remi}[thm]{Reminder}
\newtheorem{rem}[thm]{Remark}
\newtheorem{defn}[thm]{Definition}
\newtheorem{ex}[thm]{Example}
\newtheorem{nonex}[thm]{{\bf NON}-Example}
\newtheorem{conj}[thm]{Conjecture}
\newtheorem{Exercise}[thm]{Exercise}
\newtheorem{Code}[thm]{Pseudocode}
\usepackage{datetime2}
\usepackage{graphicx}
\usepackage{svg}
% \svgpath{{/home/vassago/LATEX/}{/home/vassago/LATEX/svgs/}}
\usepackage{svg-extract}
\usepackage{stmaryrd}
\usepackage{hyperref}
\usepackage{inconsolata}

\usepackage{tcolorbox}

\usepackage[all]{xy}
\usepackage{xypic}
\usepackage{xcolor}

\definecolor{light}{RGB}{0,100,43}
\definecolor{dark}{RGB}{130,0,100}
\definecolor{exercise}{RGB}{220,200,255}
\definecolor{textcolor}{RGB}{0, 150, 128}
\definecolor{title}{RGB}{150, 0, 128}
\definecolor{footercolor}{RGB}{0, 128, 64}
\definecolor{code}{RGB}{240,220,160}
\definecolor{codebg}{RGB}{50,50,50}
\definecolor{mathbg}{RGB}{0,10,30}
\definecolor{mathtext}{RGB}{220,220,250}
\definecolor{bg}{RGB}{32, 0, 16}

\def\exercise#1{\color{light}{\h\Exercise{\color{exercise}\scshape #1}}\color{textcolor}}
\def\code#1#2{\color{dark}{\h\Code{\h}\\[.5em]%
\color{mathtext}#1
\begin{tcolorbox}[colback=codebg]%
\color{code}%
\texttt{#2}%
\color{textcolor}%
\end{tcolorbox}\color{textcolor}}}

\def\backtick{`}
\def\curtime{\DTMdate{2023-09-21} ~\DTMtime{01:42:17}\DTMdisplayzone{-5}{00}}
\def\light#1{{\color{light}#1}}
\def\dark#1{{\color{dark}#1}}
\def\h{\hspace{1em}}

\def\signature{\href{mailto:marc@lange-data.org}{Dr. Marc Lange, marc@lange-data.org}}
\def\mailsignature{\href{mailto:marc@lange-data.org}{marc@lange-data.org}}

\def\abs#1{\left| #1 \right| }
\def\floor#1{\lfloor #1 \rfloor }

\newcommand{\lthree}{<\hspace{-5.2pt}3}
\newcommand{\E}{\mathcal{E}}
\newcommand{\DAG}{\mathcal{DAG}}
\newcommand{\Z}{\mathbb{Z}}
\newcommand{\R}{\mathbb{R}}
\newcommand{\C}{\mathbb{C}}
\renewcommand{\H}{\mathbb{H}}
\renewcommand{\k}{\mathbb{k}}
\renewcommand{\P}{\mathfrak{P}}
\newcommand{\Q}{\mathbb{Q}}
\newcommand{\N}{\mathbb{N}}
\renewcommand{\S}{\mathbb{S}}
\newcommand{\D}{\mathbb{D}}
\newcommand{\LL}{\mathcal{L}^1(\Z,\Z)}
\newcommand{\Set}{\mathrm{Set}}

\renewcommand{\dag}{\mathrm{emb}\mathcal{DAG}}


\newcommand{\s}{Set^{\phantom{\leq}}_{\phantom{+}}}
\newcommand{\spp}{Set=fSet}
\newcommand{\smp}{Set^{\leq}_{\phantom{+}}}
\newcommand{\spm}{Set^{\phantom{\leq}}_{+}}
\newcommand{\smm}{Set^{\leq}_{+}}
\newcommand{\spmpm}{Set^{\pm}_{\pm}}

\newcommand{\App}{A}
\newcommand{\Amp}{A^{\leq}_{\phantom{+}}}
\newcommand{\Apm}{A^{\phantom{\leq}}_{+}}
\newcommand{\Amm}{A^{\leq}_{+}}

\begin{document}
\defn{
    Consider the category of commutative rigs $Crig$. An object is a
    crig, that is "ring without negatives", i.e. a tuple $(A,+,\cdot)$
    with $A$ a set, maps $+\colon A\times A\rightarrow A$, $\cdot\colon A\times A\rightarrow A$,
    which are associative and distributive in the usual way as for rings.
    However $(A,+)$ and $(A,\cdot)$ are only required to be an associative and commutative composition,
    neutral elements not required.

    A morphism of crigs $\varphi\colon A \rightarrow B$ is a set map $A\rightarrow B$,
    which is additive and multiplicative morphism with respect to $+$ and $\cdot$.

    Note that just as in groups it follows, if a crig does have either an additive
    or a multiplicative neutral element, then either of them is unique. Denote the
    additive neutral element as $0$ for the crigs for which it exists and the multiplicative
    neutral element as $1$ for the crigs where it exists. Neither condition implies the other,
    but neither are essential either, since compositions can be promoted to monoids. We shall
    not need that fact fully generally here, so we will leave it at that.
}

The following observation is a triviality, but a very useful one in absence of zero and unit requirements.

\prop{
    A map of crigs maps idempotents to idempotents with respect to both addition and multiplication.
    I.e. \[\forall e\in A,\varphi\colon A\rightarrow B\colon e^2=e \Rightarrow f(e)^2=f(e). \]
}

\lem{
    If a crig $A$ has a zero element, i.e. $(A,+,0)$ is a commutative monoid,
    then the existence of a surjective crig morphism $\varphi\colon A\rightarrow B$
    for an arbitrary crig $B$, implies $B$ has a zero too.
    \begin{proof}
    Surjectivity implies that for each test case $0+b$ there is a preimage $\bar b$ witnessing $0+\bar b = \bar b$ and
    implying $0+b=b$ in $B$.
    \end{proof}
}

\lem{
    Surjections imply multiplicative units just the same way.
}

Cancellation properties eventually obtain zero and one as consequences rather than choices in the crig sets we are interested in.
\defn{
    An associative and commutative composition $(A,+)$ is said to {\bf satisfy cancellation}, if
    \[\forall a,b\in A\colon \exists k \in A\colon a+k=b+k \Rightarrow a=b.\]
    That is, no two elements share an additive translate by the same element.
    This can be true multiplicatively in a crig, too, for the monoid $(A\setminus\{0\},\cdot,1)$, because
    $0a=0~\forall a\in A$.

    A commutative monoid $(A,+,0)$ with the cancellation property we call {\bf strongly non-negative},
    if \[ \forall a,b\in A\colon a+b=0 \Rightarrow a=b=0. \]
    That is, for each element in $A$ its negative is not already an element of A, unless the element is the zero element.
}

\lem{
    The trivial crigs $\emptyset,0$ both satisfy cancellation vacuously.
    The trivial crig $0$ satisfies strong non-negativity with respect to both addition and multiplication vacuously, the emptyset
    fails in providing a neutral element with which to even state the condition.

    In any cancellation crig $A$ we have $0=0\cdot1$, hence $0=0\cdot a$ for all $a\in A$, by $(A,\cdot,1)$ being a monoid.
    \begin{proof}
    The first two observations are logical trivialities, for the last statement:
    Assume $A \supset \{0,1,a,b\}$ not all of them necessarily distinct, but not all of them equal by assumption that
    one can choose two distinct elements in $A$.
    Note that it follows by distributivity in a crig:
    \[ 0\cdot1 = (0+0)\cdot1 = 0\cdot1+0\cdot1 \]
    which by cancellation and existence of $0$ implies:
    \[ 0 = 0\cdot1\]
    since both have the same image after translation with $0\cdot 1$. Since $1$ is a unit we can write any a as $1\cdot a$
    and by associativity all is proved. \end{proof}
}

\cor{
    Consider a crig homomorphism $\varphi\colon A\rightarrow B$ with both crigs satisfying cancellation additively.
    Then get:
    If $B$ has a zero, then:
    \[ \varphi(0)=0 \]
    if $A$ has an additively neutral element $0$. Similarly
    if $B$ has a unit, then
    \[ \varphi(1)=1 \]
    if $A$ has a multiplicatively neutral element $1$.

    In addition, if $\varphi$ is surjective, the existence of a $0$ or $1$ in $A$ implies one in $B$.
    \begin{proof}
    Copy the proof above with all the needed variations to get the result.
    \end{proof}
}

\rem{
    One can stumble into the strongly non-negative property in various ways. It could also be considered
    a very strong form of torsion-freeness. In case of $\mathbb{N}$ both perspectives coincide, so it
    does not matter for our study.

    I do not expect this property to be of much use beyond $\mathbb{N}$ and its polynomial crigs, but
    bipermutative rig categories with objects $\mathbb{N}$ come up easily in $K$-theory constructions,
    so that case is quite enough motivation to abstract away into an important property like this.
}

\prop{
    There is a chain of crigs of natural numbers $\N_{\geq2} \subset \N_{\geq1} \subset \N_{\geq0}=:\N_0 \subset \Z$, and
    there is no crig map in the reverse direction for any of these inclusions apart from the zero map in the last step.
    \begin{proof}

    \end{proof}
}


\rem{
    Since the existence of negatives from the onset of a problem is the exception
    rather than the rule in this text, we shall rarely refer to groups as such. Specifically
    only commutative groups feature as completions or other monoid quotients which just
    happen to have additive inverses.
}

\ex{
    THE example of study in this text are the natural numbers $\mathbb{N}$. Note that
    the definition left a tiny amount of wiggle room by not requiring a zero element, hence
    the set of positive natural numbers $\mathbb{N}_{>0}=\{1,\ldots,n,n+1,\ldots\}$ has
    a unique natural crig morphism to every other crig. It is given by unitality
    and induction over the fact that each natural number greater or equal to $2$
    can be written as a sum of two smaller numbers. In particular the natural numbers
    including zero $\mathbb{N}=\mathbb{N_0}$ have a unique crig morphism $\mathbb{N}_{>0} \rightarrow \mathbb{N}_0$,
    but there is no crig map in the reverse direction: Any image of zero $f0$ would have to
    satisfy $f(0)+f(0) = f(0+0) = f(0)$, but this is not possible in positive natural numbers.
}

The specific argument identifies a useful property the natural numbers but many other crigs
enjoy too, cancellation.

\prop{
    Each crig $A$ has an associated crig of polynomials in one variable $A\lbrack X\rbrack$, satisfying the
    usual universal property: There is a crig inclusion $A\rightarrow A\lbrack X\rbrack$,
    and for each crig morphism $\varphi\colon A\rightarrow B$ with target crig $B$
    and each target element $b\in B$ there exists a unique extension of $\varphi$ given
    by "evaluating $X$ at $b$":
    \[
    \xymatrix{
    A\ar[dr]_\varphi \ar[r]&A\lbrack X\rbrack\ar[d]^{ev_b}\\&B.
    }
    \]
    \begin{proof}

    \end{proof}
}

\end{document}